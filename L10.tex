\documentclass[10pt]{beamer}

%\documentclass[10pt, handout]{beamer}
\setbeameroption{show notes}

%\documentclass[10pt, a4paper]{article}
%\usepackage{beamerarticle}




\mode<article>{
	
	\usepackage{hyperref}
	
}
\mode<presentation>{
	
	\usetheme{Antibes}
	\usefonttheme{professionalfonts} 
	\usefonttheme{serif} % default family is serif
	
	%\usecolortheme{spruce} %зеленая, плохой цвет в заголовках 
	%\usecolortheme{albatross} %синяя, пхоло виден черный цвет
	
}

\newcommand{\MP}[1]{\mode<presentation>{#1} }
\newcommand{\MA}[1]{\mode<article>{#1} }

\newcommand{\ABS}[1]{\left| #1 \right|}
%\newcommand{\ABS}[1]{\mid #1 \mid}

\newcommand{\HREF}[2]{{\color{blue}\underline{\href{#1}{#2}}}}

\setbeamertemplate{caption}[numbered]


%\usepackage[T2A]{fontenc}
%\usepackage[utf8]{inputenc}
%\usepackage[russian]{babel}
%\usepackage{amsmath} %математические формулы



\usepackage{ifthen}

\usepackage{tikz}
\usetikzlibrary{arrows.meta}
\usetikzlibrary{calc}
\usetikzlibrary{decorations}
\usetikzlibrary{decorations.pathreplacing}
\newcommand{\rememb}[1]{\tikz[remember picture,baseline=-0.5ex]{\draw node[inner sep=0pt, outer sep=0pt] (#1){\strut};}}



\usepackage{fp}
\usepackage{tikz-3dplot}
\usepackage{environ}
\usepackage{animate}





\usepackage{xcolor}
%\usepackage[left=20mm,right=20mm,top=20mm,bottom=20mm,a4paper]{geometry} %поля

\usepackage{amsmath} %математические формулы
%\usepackage{amsfonts} %математические шрифты


\usepackage[e]{esvect}  %Красивая стрелочка вектора
%\let\oldvv\vv
\newcommand{\VV}[1]{\vv{#1\mathstrut}}



\usepackage{graphicx} %работа с каритнками


%\usepackage{multimedia}
%\usepackage{movie15}

%Для XeLatex/+
\usepackage{polyglossia}
\setdefaultlanguage{russian}
\setotherlanguage{english}
%\setkeys{russian}{babelshorthands=true} 


\usepackage{fontspec}

\setmainfont{Times New Roman} [Script=Cyrillic, Mapping=tex-text,]
\setsansfont{Arial} [Script=Cyrillic, Mapping=tex-text,]
%\setmonofont{Courier New} [Script=Cyrillic, Mapping=tex-text,]
\newfontfamily{\cyrillicfonttt}{Courier New}


%\usepackage{unicode-math}
%\setmathfont{TeX Gyre Termes Math}

%\setmainfont{CMU Serif}[Script=Cyrillic, Mapping=tex-text,]
%\setsansfont{CMU Sans Serif}[Script=Cyrillic, Mapping=tex-text,]
%\setmonofont{CMU Typewriter Text}[Script=Cyrillic, Mapping=tex-text,]


%-----------------


%\usepackage{caption}
%\DeclareCaptionLabelSeparator{dot}{~---~}            %Разделитель номер рисунка
%\captionsetup[figure]{justification=centering,labelsep=dot, format=plain}                        %Подпись рис. центр
%\captionsetup[table]{justification=raggedleft,labelsep=dot, format=plain, singlelinecheck=false} %Подпись табл. слева
%\captionsetup[lstlisting]{justification=raggedleft,labelsep=dot, format=plain, singlelinecheck=false}                     %Подпись рис. центр

\usepackage{indentfirst} %отступ первой строки


\usepackage[svgnames]{xcolor}


\usepackage{hyperref}

%\usepackage{showframe}


%\usepackage{tikz}

%\usepackage[hidelinks]{hyperref}%ссылки внутри документа \ref


\setlength\abovecaptionskip{-2pt}
%\setlength\belowcaptionskip{-14pt}

\setbeamerfont{caption}{size=\scriptsize}


\def\sectionname{Раздел}
\def\subsectionname{Подраздел}


\newcommand{\TC}[3]
{
	
	
	\begin{columns}
		\begin{column}{#1\textwidth}
			#2
		\end{column}
		\begin{column}{\fpeval{1-#1}\textwidth}
			#3
		\end{column}
	\end{columns}
}

\newcommand{\TCT}[3]
{
	
	\begin{columns}[T]
		\begin{column}{#1\textwidth}
			#2
		\end{column}
		\begin{column}{\fpeval{1-#1}\textwidth}
			#3
		\end{column}
	\end{columns}
}


\newcommand{\FRAME}[2]{
	\begin{frame}
		\frametitle{#1}
		#2
	\end{frame}
}

\newcommand{\FIG}[3]
{
	\begin{figure}
		\centering
		\includegraphics[width=#3]{#1}
		\caption{#2}
	\end{figure}
}

\newcommand{\vect}[1]{\overrightarrow{#1}}


\usepackage{qrcode}

\newcommand{\LECADDR}{https://clck.ru/3D3Efj}


\usepackage{newfile}

\edef\LectionNumber{0}
\edef\LectionTheme{0}

\let\oldsection\section
\let\oldsubsection\subsection


\AtBeginDocument
{
	\newoutputstream{CONTENT}
	\openoutputfile{\LectionNumber .gvr}{CONTENT}
	
	\expandafter\addtostream{CONTENT}{\noindent\textbf{\Large Лекция \LectionNumber~---~\LectionTheme}\unexpanded{\setcounter{SEC}{0}}\par}
}

\renewcommand{\section}[1]{
	\oldsection{#1}
	\expandafter\addtostream{CONTENT}{\noindent\hspace{2ex}\unexpanded{\hbox{\large\stepcounter{SEC}\theSEC ~ #1}}\par}
}

\renewcommand{\subsection}[1]{
	\oldsubsection{#1}
	\expandafter\addtostream{CONTENT}{\noindent\hspace{6ex}\unexpanded{\stepcounter{SUB}\theSUB ~ #1}\par}
}


%\renewcommand{\section}[1]{\MMM{#1}}

%\edef\subsection#1
{
	%\noexpand\subsection{#1}
	%
}


\newfontfamily\dnifamily[Scale = 0.795]{DniFont.TTF}

\newcommand{\dni}[1]{%
	{\dnifamily%
		\ifthenelse{#1=0}{0}{}%
		\ifthenelse{#1=1}{1}{}%
		\ifthenelse{#1=2}{2}{}%
		\ifthenelse{#1=3}{3}{}%
		\ifthenelse{#1=4}{4}{}%
		\ifthenelse{#1=5}{5}{}%
		\ifthenelse{#1=6}{6}{}%
		\ifthenelse{#1=7}{7}{}%
		\ifthenelse{#1=8}{8}{}%
		\ifthenelse{#1=9}{9}{}%
		\ifthenelse{#1=10}{)}{}%
		\ifthenelse{#1=11}{!}{}%
		\ifthenelse{#1=12}{@}{}%
		\ifthenelse{#1=13}{\#}{}%
		\ifthenelse{#1=14}{\$}{}%
		\ifthenelse{#1=15}{\%}{}%
		\ifthenelse{#1=16}{\^{}}{}%
		\ifthenelse{#1=17}{\&}{}%
		\ifthenelse{#1=18}{*}{}%
		\ifthenelse{#1=19}{(}{}%
		\ifthenelse{#1=20}{[}{}%
		\ifthenelse{#1=21}{]}{}%
		\ifthenelse{#1=22}{\textbackslash{}}{}%
		\ifthenelse{#1=23}{\{}{}%
		\ifthenelse{#1=24}{\}}{}%
		\ifthenelse{#1=25}{|}{}}%
}%

\newcommand{\toDni}[1]{%	
	\ifthenelse{#1=0}{}{%
		 \ifthenelse{#1=25}{%
		 	\expandafter\dni{#1}}{%
		 	\expandafter\toDni{\fpeval{floor(#1/25)}}%
		 \expandafter\dni{\fpeval{(#1/25 - floor(#1/25))/0.04}}}}%
}%



\newcommand{\Strut}{{\Large\strut}}

\newcommand\scb[1]{\left( #1 \right)}

\newcommand{\LINK}[2]{%
	\qrcode[height=1cm]{#1}\  \HREF{#1}{\parbox{0.8\textwidth}{#2}} \\[0.5em]
}

\NewDocumentCommand{\lecdni}{}{\toDni{\LectionNumber}}
\author{Гаврилов Андрей Геннадьевич}
\newcommand{\regals}{кандидат технических наук, доцент}
\institute{Кафедра Информационных технологий и вычислительных систем \\МГТУ~<<СТАНКИН>>}
\lecture{История компьютерной графики}{kghistory}\subtitle{Компьютерная графика}


\makeatletter
\newcommand*{\overlaynumber}{\number\beamer@slideinframe}
\makeatother



\usepackage{cprotect}

\newcommand{\QRFRAME}{%
    \begin{frame}[plain, noframenumbering]    	
	
	\centering
	Трансляция презентации (во время очных лекций)    
	
	~
	
	{\Large \ttfamily  https://clck.ru/3D3Efj  }
	
	~
	
	\tikz\node[inner sep=0pt,rounded corners=5mm, clip]{\qrcode[height=0.45\textwidth]{\LECADDR}}; 
	
	~	
	{\small
	При просмотре презентации в PDF для отображения анимаций на слайдах необходимо использовать Acrobat Reader, KDE Okular, PDF-XChange, Foxit Reader, браузер Firefox. Для браузеров на движке Chrome (Edge, Яндекс, Opera,~\dots) необходимо использовать \HREF{https://chromewebstore.google.com/detail/pdf-viewer/oemmndcbldboiebfnladdacbdfmadadm?hl=ru&utm_source=ext_sidebar}{PDF.js} c опцией <<Enable active content (JavaScript) in PDFs>>. }
	
	\end{frame}%
}

\newcommand{\IG}[2][1]{\includegraphics[width=#1\textwidth]{#2}}



\graphicspath{{Images/}{Images/\jobname/}}

\date{\today}



\renewcommand{\LectionNumber}{10}
\renewcommand{\LectionTheme}{Наложение текстур}
\title{Лекция \lecdni \\ \LectionTheme}
\subtitle{Компьютерная графика}



%\usepackage{standalone}

\setbeamersize
{
	text margin left=0.5cm,
	text margin right=0.5cm
}

\usepackage{comment}


%	\transduration{2}
%   \transfade

 \begin{document}
 		 
	\makeatletter
\defbeamertemplate*{title page}{my theme}
{
	
	\hfill
	
	\begin{beamercolorbox}[wd=.9\paperwidth,center,]{title}%
		
	\end{beamercolorbox}%	
	
	\vbox to 1em {}
	
	\begin{beamercolorbox}[wd=.9\paperwidth,center,]{title}%
		\usebeamerfont{subtitle}%
		\hfill
		
		\insertsubtitle
		
		\usebeamerfont{title}%
		\inserttitle{} \\[0.5em]
		
		
		
	\end{beamercolorbox}%	
	\hfill\hfill
	
	\begin{beamercolorbox}[wd=.9\paperwidth,center,]{}%
		\usebeamerfont{author}%
		\hfill \\[0.5em]
		\insertauthor{} \\[0.5em]
		\regals
		    
		\vbox to 1em{}
		\usebeamerfont{institute}%
		\insertinstitute {}
		
		\vbox to 1em{}			
		{\; }\insertshortdate{}
		
	\end{beamercolorbox}%	
	\hfill\hfill
	
	\vbox to 5em{}
	
	
}
\defbeamertemplate*{footline}{my theme}{
	\leavevmode%
	\hbox{%
		\begin{beamercolorbox}[wd=.25\paperwidth,ht=3.25ex,dp=0ex,center,sep=1pt]{author in head/foot}%
			\usebeamerfont{author in head/foot}%
			\insertauthor 
			\beamer@ifempty{\insertshortinstitute}{}
		\end{beamercolorbox}%
		\begin{beamercolorbox}[wd=.65\paperwidth,ht=3.25ex,dp=0ex,center,sep=1pt]{title in head/foot}%
			\usebeamerfont{title in head/foot}\insertshortinstitute
		\end{beamercolorbox}%
		\begin{beamercolorbox}[wd=.1\paperwidth,ht=3.25ex,dp=0ex,center, sep=0.5pt]{date in head/foot}%
			\usebeamerfont{date in head/foot}
			\footnotesize \expandafter\toDni{\insertframenumber} / \expandafter\toDni{\inserttotalframenumber}
	\end{beamercolorbox}}%
}



\makeatother






%float page top aligment
\makeatletter
\setlength{\@fptop}{0pt}
\setlength{\@fpbot}{0pt plus 1fil}
\makeatother

\newcommand \abs[1] {\left| #1 \right|}

\everymath{\displaystyle}

    
    \QRFRAME	
	

	\frame{\maketitle}

	
	\begin{frame}{План лекции}
		\tableofcontents
	\end{frame}
	
	\begin{frame}
		\TC{0.6}
		{
			\IG{widen_1220x0-768x377.jpg}
		}
		{
			\IG{Screenshot_5.png}
		}
		
		\IG{watermelon.png}
	\end{frame}
	
	\section{Виды текстур}
	
	\begin{frame}{Пример использования различных видов текстур}
		
	{
		\centering
		\IG[0.7]{image001.png}
		
		А – карта диффузного отражения, Б – карта нормалей, В – карта смещений.
	}
		
	\end{frame}	
	
	\section{Текстурные координаты}
	
	\begin{frame} {Текстурные координаты}
		
		\centering
		\IG[0.7]{intro2.jpg}
		
	\end{frame}
	
	\begin{frame}{А что за пределами текстурных координат?}
		
		\def\p{0.33}
		\centering
		\IG[\p]{Wrap-Addressing-Mode1}
		\IG[\p]{Mirror-Addressing-Mode1}
		\IG[\p]{Clamp-Addressing-Mode1}
		\IG[\p]{Border-Addressing-Mode}
		
		
	\end{frame}
	
	\begin{frame}{Наложение текстур}
		
		\IG{textureoverview.jpg}
		
		
	\end{frame}
	
	\begin{frame}{Наложение текстур}
		\centering\IG[0.8]{Screenshot 2024-09-25 204444.png}		
	\end{frame}
	
     \begin{frame}{Наложение текстур на объекты}
     	\TC{0.5}
     	{
     		\IG{dice_unity.png}
     	}
     	{
     		\IG{dice_unwrap.png}
     	}
     	
     \end{frame}
     
     \begin{frame}{UV развертка}
     
     \centering\IG[0.9]{lol_a.jpg}
     	
     \end{frame}
     
	 \section{Фильтрация текстур}
     
     \begin{frame}{Mip-mapping}
     	
     	\centering{Multum In Parvo — «много в малом»}
     	
     	~
     	
     	\TC{0.4}
     	{
     		\IG{Original-image-left-and-quadtree-MIP-map-of-the-same-image-right-Empty-tiles-are-not.png}
     	}{
     		\IG{786px-Mip_colors_chadwick.jpg}
     	}
     	
     	
     \end{frame}
     
     \begin{frame}{Mip-mapping }
     	
     	
     	
     	\TC{0.5}
     	{
     		\IG{mip4.png}
     	}
     	{
     		\IG{mip2.jpg}
     	}
     	
     	\centering
     	
     	\IG[0.7]{Screenshot 2024-09-25 224251.png}
     	
     	
     \end{frame}
     
     \begin{frame}{Отображение текстуры на экране}
     	
     	\centering\fbox{\IG[0.8]{c3e96f30ec61cdb9f6f166de3299115c.png}}
     	
     	Как совместить тексили и пиксели?
     	
     \end{frame}
     
     \begin{frame}{Фильтрация текстур}
     	
     	
     \centering\IG[0.85]{33065527_8464b3b085d2d9561e4298209210dbff_800.jpg}
     	
    \end{frame}	
     

     
      \begin{frame}{Point Sample}
     	
     	\TC{0.6}
     	{
     		\includegraphics[page=1]{filtering.pdf}
     	}
     	{
     		Дано: текстура $3 \times 3$,\\ где
     		$T_{ij}$ --- цвет текселя $(i,j)$
     		
     		Найти: цвет пикселя $T_{uv}, u,v \in \mathbb R$
     		
     		~ \pause
     		
     		$u := \mathrm{int}(u) $
     		
     		$v := \mathrm{int}(v) $
     		
     		$\mathrm{return} \ T_{uv}$
     		
     	}
     	
     \end{frame}	
     	
     	
      \begin{frame}{Point Sample}
     		
     		\TC{0.5}
     		{
     			\IG{brick-pointsample.png}
     		}{
     			\IG{grid-pointsample.png}
     		}
     \end{frame}	
     
      \begin{frame}{Линейная фильтрация}
     	
     	\TC{0.6}
     	{
     		\only<1>{\includegraphics[page=1]{filtering.pdf}}%
     		\only<2->{\includegraphics[page=2]{filtering.pdf}}%
     	}
     	{
     		Дано: текстура $3 \times 3$,\\ где
     		$T_{ij}$ --- цвет текселя $(i,j)$
     		
     		Найти: цвет пикселя $T_{uv}, u,v \in \mathbb R$
     		
     		~ \pause
     		
     		$x := \mathrm{int}(u) $
     		
     		$y := \mathrm{int}(v) $
     		
     		~
     		
     	
     		
     		$\alpha := u-0.5-x $
     		
     		~
     		
     		$T_{uv}=\alpha T_{xy}+(1-\alpha)T_{(x+1)y}$
     		
     		~
     		
     		$\mathrm{return} \ T_{uv}$
     		
     	}
     	
     \end{frame}	
     
         \begin{frame}{Билинейная фильтрация}
     	
     	\TC{0.6}
     	{
     		\only<1>{\includegraphics[page=1]{filtering.pdf}}%
     		\only<2->{\includegraphics[page=3]{filtering.pdf}}%
     	}
     	{
     		Дано: текстура $3 \times 3$,\\ где
     		$T_{ij}$ --- цвет текселя $(i,j)$
     		
     		Найти: цвет пикселя $T_{uv}, u,v \in \mathbb R$
     		
     		~ \pause
     		
     	    $x := \mathrm{int}(u) $
     		
     		$y := \mathrm{int}(v) $
     		
     		
     	
     		
     		~
     		
     			$\alpha := u-0.5-x $
     		$\beta := v-0.5- y $

     		
     		~ \pause
	     	
	     	$T_{uv}=$
	     	
     		$\begin{array}{ll}
     			        &	 {}=\alpha\beta T_{xy} + {} \\
     					&    {}+(1-\alpha)\beta T_{(x+1)y} + {} \\
     					&	 {}+\alpha(1-\beta) T_{x(y+1)} + {} \\
     					&	 {}+(1-\alpha)(1-\beta) T_{(x+1)(y+1)}
     		\end{array}$
     		
     		~
     		
     		$\mathrm{return} \ T_{uv}$
     		
     	}
     	
     \end{frame}	
     
     
     \begin{frame}{Билинейная фильтрация}
     	
     	\centering
     	
     	\IG[0.32]{bilinear-fig7a}
     	\IG[0.32]{bilinear-fig7b} 	
     	\IG[0.32]{bilinear-fig7c}
     	
     \end{frame}
     
     
     \begin{frame}{Билинейная фильтрация}
     	\centering\IG[0.5]{Screenshot 2024-09-25 223349}
     \end{frame}
     
     \begin{frame}{Трилинейная фильтрация}
     	
     	\centering
     	
     	\IG[0.6]{trilinearfilter}
     	
     \end{frame}
     
          
     \begin{frame}{Трилинейная фильтрация}
     	
     	\centering
     	
     	\IG[0.9]{Screenshot 2024-09-25 223603}
     	
     \end{frame}
     
     
     \begin{frame}{Анизотропная фильтрация}
     	\IG{Mipmap-Filtering}
     	
     	
     	
     	
     \end{frame}
     
     
     \begin{frame} {Анизотропная фильтрация}
     	
     	
     	\begin{center}
     		\centering 	\IG[0.6]{Screenshot 2024-09-26 063729}
     	\end{center}
     	     	
     	\HREF{https://www.reddit.com/r/Shadron/comments/6p0764/comparison_of_texture_filtering_modes/}{Примеры}
     \end{frame}
 
\begin{comment}
	
\end{comment}
 

\end{document}