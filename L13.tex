\documentclass[10pt]{beamer}

%\documentclass[10pt, handout]{beamer}
\setbeameroption{show notes}

%\documentclass[10pt, a4paper]{article}
%\usepackage{beamerarticle}




\mode<article>{
	
	\usepackage{hyperref}
	
}
\mode<presentation>{
	
	\usetheme{Antibes}
	\usefonttheme{professionalfonts} 
	\usefonttheme{serif} % default family is serif
	
	%\usecolortheme{spruce} %зеленая, плохой цвет в заголовках 
	%\usecolortheme{albatross} %синяя, пхоло виден черный цвет
	
}

\newcommand{\MP}[1]{\mode<presentation>{#1} }
\newcommand{\MA}[1]{\mode<article>{#1} }

\newcommand{\ABS}[1]{\left| #1 \right|}
%\newcommand{\ABS}[1]{\mid #1 \mid}

\newcommand{\HREF}[2]{{\color{blue}\underline{\href{#1}{#2}}}}

\setbeamertemplate{caption}[numbered]


%\usepackage[T2A]{fontenc}
%\usepackage[utf8]{inputenc}
%\usepackage[russian]{babel}
%\usepackage{amsmath} %математические формулы



\usepackage{ifthen}

\usepackage{tikz}
\usetikzlibrary{arrows.meta}
\usetikzlibrary{calc}
\usetikzlibrary{decorations}
\usetikzlibrary{decorations.pathreplacing}
\newcommand{\rememb}[1]{\tikz[remember picture,baseline=-0.5ex]{\draw node[inner sep=0pt, outer sep=0pt] (#1){\strut};}}



\usepackage{fp}
\usepackage{tikz-3dplot}
\usepackage{environ}
\usepackage{animate}





\usepackage{xcolor}
%\usepackage[left=20mm,right=20mm,top=20mm,bottom=20mm,a4paper]{geometry} %поля

\usepackage{amsmath} %математические формулы
%\usepackage{amsfonts} %математические шрифты


\usepackage[e]{esvect}  %Красивая стрелочка вектора
%\let\oldvv\vv
\newcommand{\VV}[1]{\vv{#1\mathstrut}}



\usepackage{graphicx} %работа с каритнками


%\usepackage{multimedia}
%\usepackage{movie15}

%Для XeLatex/+
\usepackage{polyglossia}
\setdefaultlanguage{russian}
\setotherlanguage{english}
%\setkeys{russian}{babelshorthands=true} 


\usepackage{fontspec}

\setmainfont{Times New Roman} [Script=Cyrillic, Mapping=tex-text,]
\setsansfont{Arial} [Script=Cyrillic, Mapping=tex-text,]
%\setmonofont{Courier New} [Script=Cyrillic, Mapping=tex-text,]
\newfontfamily{\cyrillicfonttt}{Courier New}


%\usepackage{unicode-math}
%\setmathfont{TeX Gyre Termes Math}

%\setmainfont{CMU Serif}[Script=Cyrillic, Mapping=tex-text,]
%\setsansfont{CMU Sans Serif}[Script=Cyrillic, Mapping=tex-text,]
%\setmonofont{CMU Typewriter Text}[Script=Cyrillic, Mapping=tex-text,]


%-----------------


%\usepackage{caption}
%\DeclareCaptionLabelSeparator{dot}{~---~}            %Разделитель номер рисунка
%\captionsetup[figure]{justification=centering,labelsep=dot, format=plain}                        %Подпись рис. центр
%\captionsetup[table]{justification=raggedleft,labelsep=dot, format=plain, singlelinecheck=false} %Подпись табл. слева
%\captionsetup[lstlisting]{justification=raggedleft,labelsep=dot, format=plain, singlelinecheck=false}                     %Подпись рис. центр

\usepackage{indentfirst} %отступ первой строки


\usepackage[svgnames]{xcolor}


\usepackage{hyperref}

%\usepackage{showframe}


%\usepackage{tikz}

%\usepackage[hidelinks]{hyperref}%ссылки внутри документа \ref


\setlength\abovecaptionskip{-2pt}
%\setlength\belowcaptionskip{-14pt}

\setbeamerfont{caption}{size=\scriptsize}


\def\sectionname{Раздел}
\def\subsectionname{Подраздел}


\newcommand{\TC}[3]
{
	
	
	\begin{columns}
		\begin{column}{#1\textwidth}
			#2
		\end{column}
		\begin{column}{\fpeval{1-#1}\textwidth}
			#3
		\end{column}
	\end{columns}
}

\newcommand{\TCT}[3]
{
	
	\begin{columns}[T]
		\begin{column}{#1\textwidth}
			#2
		\end{column}
		\begin{column}{\fpeval{1-#1}\textwidth}
			#3
		\end{column}
	\end{columns}
}


\newcommand{\FRAME}[2]{
	\begin{frame}
		\frametitle{#1}
		#2
	\end{frame}
}

\newcommand{\FIG}[3]
{
	\begin{figure}
		\centering
		\includegraphics[width=#3]{#1}
		\caption{#2}
	\end{figure}
}

\newcommand{\vect}[1]{\overrightarrow{#1}}


\usepackage{qrcode}

\newcommand{\LECADDR}{https://clck.ru/3D3Efj}


\usepackage{newfile}

\edef\LectionNumber{0}
\edef\LectionTheme{0}

\let\oldsection\section
\let\oldsubsection\subsection


\AtBeginDocument
{
	\newoutputstream{CONTENT}
	\openoutputfile{\LectionNumber .gvr}{CONTENT}
	
	\expandafter\addtostream{CONTENT}{\noindent\textbf{\Large Лекция \LectionNumber~---~\LectionTheme}\unexpanded{\setcounter{SEC}{0}}\par}
}

\renewcommand{\section}[1]{
	\oldsection{#1}
	\expandafter\addtostream{CONTENT}{\noindent\hspace{2ex}\unexpanded{\hbox{\large\stepcounter{SEC}\theSEC ~ #1}}\par}
}

\renewcommand{\subsection}[1]{
	\oldsubsection{#1}
	\expandafter\addtostream{CONTENT}{\noindent\hspace{6ex}\unexpanded{\stepcounter{SUB}\theSUB ~ #1}\par}
}


%\renewcommand{\section}[1]{\MMM{#1}}

%\edef\subsection#1
{
	%\noexpand\subsection{#1}
	%
}


\newfontfamily\dnifamily[Scale = 0.795]{DniFont.TTF}

\newcommand{\dni}[1]{%
	{\dnifamily%
		\ifthenelse{#1=0}{0}{}%
		\ifthenelse{#1=1}{1}{}%
		\ifthenelse{#1=2}{2}{}%
		\ifthenelse{#1=3}{3}{}%
		\ifthenelse{#1=4}{4}{}%
		\ifthenelse{#1=5}{5}{}%
		\ifthenelse{#1=6}{6}{}%
		\ifthenelse{#1=7}{7}{}%
		\ifthenelse{#1=8}{8}{}%
		\ifthenelse{#1=9}{9}{}%
		\ifthenelse{#1=10}{)}{}%
		\ifthenelse{#1=11}{!}{}%
		\ifthenelse{#1=12}{@}{}%
		\ifthenelse{#1=13}{\#}{}%
		\ifthenelse{#1=14}{\$}{}%
		\ifthenelse{#1=15}{\%}{}%
		\ifthenelse{#1=16}{\^{}}{}%
		\ifthenelse{#1=17}{\&}{}%
		\ifthenelse{#1=18}{*}{}%
		\ifthenelse{#1=19}{(}{}%
		\ifthenelse{#1=20}{[}{}%
		\ifthenelse{#1=21}{]}{}%
		\ifthenelse{#1=22}{\textbackslash{}}{}%
		\ifthenelse{#1=23}{\{}{}%
		\ifthenelse{#1=24}{\}}{}%
		\ifthenelse{#1=25}{|}{}}%
}%

\newcommand{\toDni}[1]{%	
	\ifthenelse{#1=0}{}{%
		 \ifthenelse{#1=25}{%
		 	\expandafter\dni{#1}}{%
		 	\expandafter\toDni{\fpeval{floor(#1/25)}}%
		 \expandafter\dni{\fpeval{(#1/25 - floor(#1/25))/0.04}}}}%
}%



\newcommand{\Strut}{{\Large\strut}}

\newcommand\scb[1]{\left( #1 \right)}

\newcommand{\LINK}[2]{%
	\qrcode[height=1cm]{#1}\  \HREF{#1}{\parbox{0.8\textwidth}{#2}} \\[0.5em]
}

\NewDocumentCommand{\lecdni}{}{\toDni{\LectionNumber}}
\author{Гаврилов Андрей Геннадьевич}
\newcommand{\regals}{кандидат технических наук, доцент}
\institute{Кафедра Информационных технологий и вычислительных систем \\МГТУ~<<СТАНКИН>>}
\lecture{История компьютерной графики}{kghistory}\subtitle{Компьютерная графика}


\makeatletter
\newcommand*{\overlaynumber}{\number\beamer@slideinframe}
\makeatother



\usepackage{cprotect}

\newcommand{\QRFRAME}{%
    \begin{frame}[plain, noframenumbering]    	
	
	\centering
	Трансляция презентации (во время очных лекций)    
	
	~
	
	{\Large \ttfamily  https://clck.ru/3D3Efj  }
	
	~
	
	\tikz\node[inner sep=0pt,rounded corners=5mm, clip]{\qrcode[height=0.45\textwidth]{\LECADDR}}; 
	
	~	
	{\small
	При просмотре презентации в PDF для отображения анимаций на слайдах необходимо использовать Acrobat Reader, KDE Okular, PDF-XChange, Foxit Reader, браузер Firefox. Для браузеров на движке Chrome (Edge, Яндекс, Opera,~\dots) необходимо использовать \HREF{https://chromewebstore.google.com/detail/pdf-viewer/oemmndcbldboiebfnladdacbdfmadadm?hl=ru&utm_source=ext_sidebar}{PDF.js} c опцией <<Enable active content (JavaScript) in PDFs>>. }
	
	\end{frame}%
}

\newcommand{\IG}[2][1]{\includegraphics[width=#1\textwidth]{#2}}




\usepackage{media9}

\graphicspath{{Images/}{Images/\jobname/}}

\date{\today}


\renewcommand{\LectionNumber}{13}
\renewcommand{\LectionTheme}{рациональные кривые и поверхности}
\title{Лекция \lecdni \\ \LectionTheme}
\subtitle{Компьютерная графика}



%\usepackage{standalone}

\setbeamersize
{
	text margin left=0.5cm,
	text margin right=0.5cm
}

\usepackage{comment}


%	\transduration{2}
%   \transfade

\begin{document}
 		 
	\makeatletter
\defbeamertemplate*{title page}{my theme}
{
	
	\hfill
	
	\begin{beamercolorbox}[wd=.9\paperwidth,center,]{title}%
		
	\end{beamercolorbox}%	
	
	\vbox to 1em {}
	
	\begin{beamercolorbox}[wd=.9\paperwidth,center,]{title}%
		\usebeamerfont{subtitle}%
		\hfill
		
		\insertsubtitle
		
		\usebeamerfont{title}%
		\inserttitle{} \\[0.5em]
		
		
		
	\end{beamercolorbox}%	
	\hfill\hfill
	
	\begin{beamercolorbox}[wd=.9\paperwidth,center,]{}%
		\usebeamerfont{author}%
		\hfill \\[0.5em]
		\insertauthor{} \\[0.5em]
		\regals
		    
		\vbox to 1em{}
		\usebeamerfont{institute}%
		\insertinstitute {}
		
		\vbox to 1em{}			
		{\; }\insertshortdate{}
		
	\end{beamercolorbox}%	
	\hfill\hfill
	
	\vbox to 5em{}
	
	
}
\defbeamertemplate*{footline}{my theme}{
	\leavevmode%
	\hbox{%
		\begin{beamercolorbox}[wd=.25\paperwidth,ht=3.25ex,dp=0ex,center,sep=1pt]{author in head/foot}%
			\usebeamerfont{author in head/foot}%
			\insertauthor 
			\beamer@ifempty{\insertshortinstitute}{}
		\end{beamercolorbox}%
		\begin{beamercolorbox}[wd=.65\paperwidth,ht=3.25ex,dp=0ex,center,sep=1pt]{title in head/foot}%
			\usebeamerfont{title in head/foot}\insertshortinstitute
		\end{beamercolorbox}%
		\begin{beamercolorbox}[wd=.1\paperwidth,ht=3.25ex,dp=0ex,center, sep=0.5pt]{date in head/foot}%
			\usebeamerfont{date in head/foot}
			\footnotesize \expandafter\toDni{\insertframenumber} / \expandafter\toDni{\inserttotalframenumber}
	\end{beamercolorbox}}%
}



\makeatother






%float page top aligment
\makeatletter
\setlength{\@fptop}{0pt}
\setlength{\@fpbot}{0pt plus 1fil}
\makeatother

\newcommand \abs[1] {\left| #1 \right|}

\everymath{\displaystyle}

    
\begin{comment}
\end{comment}


    
    
    \QRFRAME	
	

	\frame{\maketitle}

	
	\begin{frame}\frametitle{План лекции}
		\tableofcontents
	\end{frame}
	




\section{Рациональные кривые}
\frame {\sectionpage}


\begin{frame}
	\frametitle<1>{Рациональные кривые}
	\frametitle<2>{Рациональные кривые живут в однородных координатах}
	
	\TC{0.55}
	{
		\only<2->{\animategraphics[autoplay,loop, nomouse, palindrome]{20}{Images/\jobname/nurbs3d}{0}{99}}
		
	}{			
		\animategraphics[autoplay,loop, nomouse, palindrome]{20}{Images/\jobname/nurbs3d}{100}{199}
		
		~
		
		\onslide<2->{%
			$A^w, B^w, C^w$ --- $(wx,wy,w)$.
			
			$A, B, C$ --- $(x,y,1)$.
			
			$A^w \to A: (wx,wy,w) \to $
			
			$\to (x,y,1) \to (x,y)$.				
			
			Если $w=0$, то $A^w = (0,0,0)$.
		}%				
	}%		
	
\end{frame} 



\begin{frame}\frametitle{Как оно работает?}
	\fontsize{8pt}{10pt}\selectfont
	\begin{center}
			\begin{tabular}{ccc}
			было в 3D & & стало в 4D \\
			$\vv P (t)=\sum_{i=0}^{n}\vv B_iN_{i,k}(t)$&  $\Rightarrow$  & $\vv {P^w} (t)=\sum_{i=0}^{n}\vv B^w_iN_{i,k}(t), $ \\
			
			$P(t), B_i \rightarrow (x,y,z)$&&$P(t)^w, B_i^w \rightarrow (wx,wy,wz,w)$
		\end{tabular}
	\end{center}
	
	\pause
	
	$\vv B^w_i = \vv B_iw_i$  $\Rightarrow$ $ P (t)=\dfrac{\displaystyle\sum_{i=0}^{n}\left[\vv B_i\rememb{n3}w_iN_{i,k}(t)\rememb{n4}\right]}{\displaystyle\rememb{n1}\sum_{i=0}^{n}\left[\vv w_iN_{i,k}(t)\right]\rememb{n2}}$
	
	\pause
	\tikz[remember picture, overlay]{\draw[red] ($(n1.south west)-(0,2ex)$) rectangle ($(n2.north east)+(0,1.9ex)$);}
	\tikz[remember picture, overlay]{\draw[red] ($(n3.south west)$) rectangle (n4.north east);}
	
	\begin{tikzpicture}[remember picture, overlay]
		\coordinate (p1) at ($(n1.west)!0.5!(n4.west)+(2cm,0)$);
		\draw (p1) node[right, red] (r)  {$R_{i,k}(t)$};
		\draw[red,->] (n2.east) -- (r);
		\draw[red,->] (n4.east) -- (r);
	\end{tikzpicture}
	
	
	\begin{center}		
		\fbox{$\vv P (t)=\sum_{i=0}^{n}\vv B_iR_{i,k}(t)$, \quad $R_{i,k} (t)=\dfrac{w_iN_{i,k}(t)}{\displaystyle\rememb{n1}\sum_{j=0}^{n} w_jN_{j,k}(t)\rememb{n2}}$}
	\end{center}

	
	
\end{frame}


\begin{frame}\frametitle{Веса рационального b-сплайна}

	
	\begin{center}
		\includegraphics[page=1]{nurbs}		
		\includegraphics[page=2]{nurbs}
	\end{center}
	

	
	\TC{0.5}
	{
		$\vv P (t)=\sum_{i=0}^{n}\vv B_iR_{i,k}(t)$
		
		$B={B_0, \dots, B_4}$
		
		$x=\{ 0,0,0,1,2,3,3,3\}$ (левый)
		
		$x= \{ 0,0,0,0,1,2,2,2,2\}$ (правый)
	}{
		{\color{Magenta} $w={1,1,3,1,1}$}
		
		{\color{blue} $w={1,1,1,1,1}$}
		
		{\color{green} $w={1,1,0.75,1,1}$}
		
		{\color{Magenta} $w={1,1,0,1,1}$}
		
	}
	
\end{frame}

\begin{frame}\frametitle{Весовые функции рационального b-сплайна}
	
	\subtitle{3го-порядка}
	
	\TC{0.45}
	{
		\animategraphics[autoplay,loop, nomouse, palindrome]{15}{Images/\jobname/nurbs_anim1}{0}{99}
		
	}{
		$P (t)=\sum_{i=0}^{4}\vv B_iR_{i,3}(t)$
		
		$x=\{ 0,0,0,1,2,3,3,3\}$ 
		
		\animategraphics[autoplay,loop, nomouse, palindrome]{15}{Images/\jobname/nurbs_weight_fun_anim1}{0}{99}
		\begin{minipage}[b]{1cm}
			{\color{red}$R_{0,3}$}\\
			{\color{green}$R_{1,3}$}\\
			{\color{blue}$R_{2,3}$}\\
			{\color{Orange}$R_{3,3}$}\\
			{\color{Magenta}$R_{4,3}$}\\
		\end{minipage}

	}	
	
\end{frame}

\begin{frame}\frametitle{Весовые функции рационального b-сплайна}
	
	\subtitle{4го-порядка}
	
	\TC{0.5}
	{
		\animategraphics[autoplay,loop, nomouse, palindrome]{15}{Images/\jobname/nurbs_anim2}{0}{99}
		
	}{
		$P (t)=\sum_{i=0}^{4}\vv B_iR_{i,4}(t)$
		
		$x= \{ 0,0,0,0,1,2,2,2,2\}$
		
		\animategraphics[autoplay,loop, nomouse, palindrome]{15}{Images/\jobname/nurbs_weight_fun_anim2}{0}{99}
		\begin{minipage}[b]{1cm}		
			{\color{red}$R_{0,4}$}\\
			{\color{green}$R_{1,4}$}\\
			{\color{blue}$R_{2,4}$}\\
			{\color{Orange}$R_{3,4}$}\\
			{\color{Magenta}$R_{4,4}$}\\
		\end{minipage}
	}	
	
\end{frame}



\section{Параметрические поверхности}
\frame {\sectionpage}

\begin{frame}\frametitle{Поверхность Безье}
	
	\TC{0.5}
	{
		\includegraphics{bsurf.pdf}
	}{
		$\VV S(u, v) = \sum_{i=0}^n \sum_{j=0}^m B_{i,n}(u) \; B_{j,m}(v) \; \VV{P}_{ij}$
	}
	
	
	
\end{frame}



\begin{frame}\frametitle{Поверхность на основе b-сплайна}
	
	\TC{0.45}
	{
		\includegraphics[page=33]{nurbs_surf_order3}
	}{
		$\VV S(u,v) = \sum_{i=0}^n \sum_{j=0}^m N_{i,p}(u)N_{j,q}(v)\VV P_{ij}$
		
		$U=\{\rememb{n1}0, \dots{}, 0\rememb{n2},\rememb{n3}\dots{},i-p,\dots\rememb{n4}, \rememb{n5}p, \dots ,p  \rememb{n6}\}$
		
		\begin{tikzpicture}[overlay, remember picture]
			
			
			\draw[decorate,decoration={brace,amplitude=5pt,mirror,raise=1ex}] (n1) -- (n2) node[below, midway, yshift=-1.5ex] {$[0,  p-1]$};
			\draw[decorate,decoration={brace,amplitude=5pt,mirror,raise=1ex}] (n3) -- (n4) node[below, midway, yshift=-1.5ex] {$[p,  n]$};
			\draw[decorate,decoration={brace,amplitude=5pt,mirror,raise=1ex}] (n5) -- (n6) node[below, midway, yshift=-1.5ex] {$[n+1,  n+p]$};
			
		\end{tikzpicture}
		
		\vspace{3ex}
		
		$V=\{\rememb{n1}0, \dots{}, 0\rememb{n2},\rememb{n3}\dots{},i-q,\dots\rememb{n4}, \rememb{n5}q, \dots ,q  \rememb{n6}\}$
		
		\begin{tikzpicture}[overlay, remember picture]
		
			\draw[decorate,decoration={brace,amplitude=5pt,mirror,raise=1ex}] (n1) -- (n2) node[below, midway, yshift=-1.5ex] {$[0,  q-1]$};
			\draw[decorate,decoration={brace,amplitude=5pt,mirror,raise=1ex}] (n3) -- (n4) node[below, midway, yshift=-1.5ex] {$[q,  n]$};
			\draw[decorate,decoration={brace,amplitude=5pt,mirror,raise=1ex}] (n5) -- (n6) node[below, midway, yshift=-1.5ex] {$[n+1,  n+q]$};
			
		\end{tikzpicture}
		
		

	}
	
	
\end{frame}

\section{Параметрические рациональные поверхности}
\frame{\sectionpage}

\begin{frame}\frametitle{NURBS-поверхность}
	
	$\VV S(u,v) = \dfrac{\sum_{i=0}^n\sum_{j=0}^mN_{i,p}(u)N_{i,q}(v)w_{ij}\VV P_{ij}}
						{\sum_{i=0}^n\sum_{j=0}^mN_{i,p}(u)N_{i,q}(v)w_{ij}}$, 
						
	
	

						
	\vspace{1ex} \pause
	
	\begin{center}
		\fbox{\parbox{0.6\textwidth}{\centering	$\VV S(u,v)=\sum_{i=0}^n\sum_{j=0}^mR_{i,j}^{p,q}(u,v)P_{i,j}$, $u,v\in [0,1]$\\%			
			$R_{i,j}^{p,q}(u,v)=\dfrac{N_{i,p}(u)N_{i,q}(v)w_{ij}}{\sum_{i=0}^n\sum_{j=0}^mN_{i,p}(u)N_{i,q}(v)w_{ij}}$%
		}}
	\end{center}
	
	\pause
	$U=\{\rememb{n1}0, \dots{}, 0\rememb{n2},\rememb{n3}\dots{},(i-p)/p,\dots\rememb{n4}, \rememb{n5}1, \dots ,1  \rememb{n6}\}$
	
	\begin{tikzpicture}[overlay, remember picture]
		
		
		\draw[decorate,decoration={brace,amplitude=5pt,mirror,raise=1ex}] (n1) -- (n2) node[below, midway, yshift=-1.5ex] {$[0,  p-1]$};
		\draw[decorate,decoration={brace,amplitude=5pt,mirror,raise=1ex}] (n3) -- (n4) node[below, midway, yshift=-1.5ex] {$[p,  n]$};
		\draw[decorate,decoration={brace,amplitude=5pt,mirror,raise=1ex}] (n5) -- (n6) node[below, midway, yshift=-1.5ex] {$[n+1,  n+p]$};
		
	\end{tikzpicture}
	
	
	
	\vspace{2ex}
	
	$V=\{\rememb{n1}0, \dots{}, 0\rememb{n2},\rememb{n3}\dots{},(i-q)/q,\dots\rememb{n4}, \rememb{n5}1, \dots ,1  \rememb{n6}\}$
	
	\begin{tikzpicture}[overlay, remember picture]
		
		\draw[decorate,decoration={brace,amplitude=5pt,mirror,raise=1ex}] (n1) -- (n2) node[below, midway, yshift=-1.5ex] {$[0,  q-1]$};
		\draw[decorate,decoration={brace,amplitude=5pt,mirror,raise=1ex}] (n3) -- (n4) node[below, midway, yshift=-1.5ex] {$[q,  n]$};
		\draw[decorate,decoration={brace,amplitude=5pt,mirror,raise=1ex}] (n5) -- (n6) node[below, midway, yshift=-1.5ex] {$[n+1,  n+q]$};
		
	\end{tikzpicture}
	

						
	
	
\end{frame}


\begin{frame}\frametitle{Поверхность пятого порядка}
	
	
	
	\TC{0.5}
	{
		\animategraphics[autoplay,loop, nomouse]{15}{Images/\jobname/nurbs_surf_order5}{0}{71}
	}{
		$\VV S(u,v)=\sum_{i=0}^4\sum_{j=0}^4R_{i,j}^{5,5}(u,v)P_{i,j}$,
		
		$\{P_{i,j}\}=P_{5\times5}$,
		
		$\{w_{i,j}\}=w_{5\times5}=\{1\}_{5\times5}$,
		
		$U=V=\{0,0,0,0,0,1,1,1,1\}$.
	}
	
\end{frame}

\begin{frame}\frametitle{Поверхность четвертого порядка}
	
	
	
	\TC{0.5}
	{
		\animategraphics[autoplay,loop, nomouse]{15}{Images/\jobname/nurbs_surf_order4}{0}{71}
	}{
		$\VV S(u,v)=\sum_{i=0}^4\sum_{j=0}^4R_{i,j}^{4,4}(u,v)P_{i,j}$,
		
		$\{P_{i,j}\}=P_{5\times5}$,
		
		$\{w_{i,j}\}=w_{5\times5}=\{1\}_{5\times5}$,
		
		$U=V=\{0,0,0,0,1/2,1,1,1,1\}$.
	}
	
\end{frame}

\begin{frame}\frametitle{Поверхность третьего порядка}
	
	
	
	\TC{0.5}
	{
		\animategraphics[autoplay,loop, nomouse]{15}{Images/\jobname/nurbs_surf_order3}{0}{71}
	}{
		$\VV S(u,v)=\sum_{i=0}^4\sum_{j=0}^4R_{i,j}^{3,3}(u,v)P_{i,j}$,
		
		$\{P_{i,j}\}=P_{5\times5}$,
		
		$\{w_{i,j}\}=w_{5\times5}=\{1\}_{5\times5}$,
		
		$U=V=\{0,0,0,1/3,2/3,1,1,1\}$.
	}
	
\end{frame} 

\begin{frame}\frametitle{Поверхность второго порядка}
	
	
	
	\TC{0.5}
	{
		\animategraphics[autoplay,loop, nomouse]{15}{Images/\jobname/nurbs_surf_order2}{0}{71}
	}{
		$\VV S(u,v)=\sum_{i=0}^4\sum_{j=0}^4R_{i,j}^{2,2}(u,v)P_{i,j}$,
		
		$\{P_{i,j}\}=P_{5\times5}$,
		
		$\{w_{i,j}\}=w_{5\times5}=\{1\}_{5\times5}$,
		
		$U=V=\{0,0,1/4,2/4,3/4,1,1\}$.
	}
	
\end{frame} 


\begin{frame}\frametitle{Изменение веса вершин}
	
	\TC{0.5}
	{
		\animategraphics[autoplay,loop, nomouse]{15}{Images/\jobname/nurbs_surf_order3_anim}{0}{71}
	}{
		$\VV S(u,v)=\sum_{i=0}^4\sum_{j=0}^4R_{i,j}^{3,3}(u,v)P_{i,j}$,
		
		$\{P_{i,j}\}=P_{5\times5}$,
		
		\animategraphics[autoplay,loop, nomouse]{15}{Images/\jobname/nurbs_surf_order3_anim}{72}{143}
		
		$U=V=\{0,0,0,1/3,2/3,1,1,1\}$.
	}
	
\end{frame}



\begin{frame}\frametitle{Поверхность втор...етьего порядка}
	
	\TC{0.5}
	{
		\animategraphics[autoplay,loop, nomouse]{15}{Images/\jobname/nurbs_surf_order2-3}{0}{71}
	}{
		$\VV S(u,v)=\sum_{i=0}^4\sum_{j=0}^4R_{i,j}^{3,2}(u,v)P_{i,j}$,
		
		$\{P_{i,j}\}=P_{5\times5}$,
		
		$\{w_{i,j}\}=w_{5\times5}=\{1\}_{5\times5}$,
		
		$U=\{0,0,0,1/3,2/3,1,1,1\}$,
		
		$V=\{0,0,1/4,2/4,3/4,1,1\}$.
	}
	
	
\end{frame}



\end{document}


