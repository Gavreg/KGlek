\documentclass[10pt]{beamer}

%\documentclass[10pt, handout]{beamer}
\setbeameroption{show notes}

%\documentclass[10pt, a4paper]{article}
%\usepackage{beamerarticle}




\mode<article>{
	
	\usepackage{hyperref}
	
}
\mode<presentation>{
	
	\usetheme{Antibes}
	\usefonttheme{professionalfonts} 
	\usefonttheme{serif} % default family is serif
	
	%\usecolortheme{spruce} %зеленая, плохой цвет в заголовках 
	%\usecolortheme{albatross} %синяя, пхоло виден черный цвет
	
}

\newcommand{\MP}[1]{\mode<presentation>{#1} }
\newcommand{\MA}[1]{\mode<article>{#1} }

\newcommand{\ABS}[1]{\left| #1 \right|}
%\newcommand{\ABS}[1]{\mid #1 \mid}

\newcommand{\HREF}[2]{{\color{blue}\underline{\href{#1}{#2}}}}

\setbeamertemplate{caption}[numbered]


%\usepackage[T2A]{fontenc}
%\usepackage[utf8]{inputenc}
%\usepackage[russian]{babel}
%\usepackage{amsmath} %математические формулы



\usepackage{ifthen}

\usepackage{tikz}
\usetikzlibrary{arrows.meta}
\usetikzlibrary{calc}
\usetikzlibrary{decorations}
\usetikzlibrary{decorations.pathreplacing}
\newcommand{\rememb}[1]{\tikz[remember picture,baseline=-0.5ex]{\draw node[inner sep=0pt, outer sep=0pt] (#1){\strut};}}



\usepackage{fp}
\usepackage{tikz-3dplot}
\usepackage{environ}
\usepackage{animate}





\usepackage{xcolor}
%\usepackage[left=20mm,right=20mm,top=20mm,bottom=20mm,a4paper]{geometry} %поля

\usepackage{amsmath} %математические формулы
%\usepackage{amsfonts} %математические шрифты


\usepackage[e]{esvect}  %Красивая стрелочка вектора
%\let\oldvv\vv
\newcommand{\VV}[1]{\vv{#1\mathstrut}}



\usepackage{graphicx} %работа с каритнками


%\usepackage{multimedia}
%\usepackage{movie15}

%Для XeLatex/+
\usepackage{polyglossia}
\setdefaultlanguage{russian}
\setotherlanguage{english}
%\setkeys{russian}{babelshorthands=true} 


\usepackage{fontspec}

\setmainfont{Times New Roman} [Script=Cyrillic, Mapping=tex-text,]
\setsansfont{Arial} [Script=Cyrillic, Mapping=tex-text,]
%\setmonofont{Courier New} [Script=Cyrillic, Mapping=tex-text,]
\newfontfamily{\cyrillicfonttt}{Courier New}


%\usepackage{unicode-math}
%\setmathfont{TeX Gyre Termes Math}

%\setmainfont{CMU Serif}[Script=Cyrillic, Mapping=tex-text,]
%\setsansfont{CMU Sans Serif}[Script=Cyrillic, Mapping=tex-text,]
%\setmonofont{CMU Typewriter Text}[Script=Cyrillic, Mapping=tex-text,]


%-----------------


%\usepackage{caption}
%\DeclareCaptionLabelSeparator{dot}{~---~}            %Разделитель номер рисунка
%\captionsetup[figure]{justification=centering,labelsep=dot, format=plain}                        %Подпись рис. центр
%\captionsetup[table]{justification=raggedleft,labelsep=dot, format=plain, singlelinecheck=false} %Подпись табл. слева
%\captionsetup[lstlisting]{justification=raggedleft,labelsep=dot, format=plain, singlelinecheck=false}                     %Подпись рис. центр

\usepackage{indentfirst} %отступ первой строки


\usepackage[svgnames]{xcolor}


\usepackage{hyperref}

%\usepackage{showframe}


%\usepackage{tikz}

%\usepackage[hidelinks]{hyperref}%ссылки внутри документа \ref


\setlength\abovecaptionskip{-2pt}
%\setlength\belowcaptionskip{-14pt}

\setbeamerfont{caption}{size=\scriptsize}


\def\sectionname{Раздел}
\def\subsectionname{Подраздел}


\newcommand{\TC}[3]
{
	
	
	\begin{columns}
		\begin{column}{#1\textwidth}
			#2
		\end{column}
		\begin{column}{\fpeval{1-#1}\textwidth}
			#3
		\end{column}
	\end{columns}
}

\newcommand{\TCT}[3]
{
	
	\begin{columns}[T]
		\begin{column}{#1\textwidth}
			#2
		\end{column}
		\begin{column}{\fpeval{1-#1}\textwidth}
			#3
		\end{column}
	\end{columns}
}


\newcommand{\FRAME}[2]{
	\begin{frame}
		\frametitle{#1}
		#2
	\end{frame}
}

\newcommand{\FIG}[3]
{
	\begin{figure}
		\centering
		\includegraphics[width=#3]{#1}
		\caption{#2}
	\end{figure}
}

\newcommand{\vect}[1]{\overrightarrow{#1}}


\usepackage{qrcode}

\newcommand{\LECADDR}{https://clck.ru/3D3Efj}


\usepackage{newfile}

\edef\LectionNumber{0}
\edef\LectionTheme{0}

\let\oldsection\section
\let\oldsubsection\subsection


\AtBeginDocument
{
	\newoutputstream{CONTENT}
	\openoutputfile{\LectionNumber .gvr}{CONTENT}
	
	\expandafter\addtostream{CONTENT}{\noindent\textbf{\Large Лекция \LectionNumber~---~\LectionTheme}\unexpanded{\setcounter{SEC}{0}}\par}
}

\renewcommand{\section}[1]{
	\oldsection{#1}
	\expandafter\addtostream{CONTENT}{\noindent\hspace{2ex}\unexpanded{\hbox{\large\stepcounter{SEC}\theSEC ~ #1}}\par}
}

\renewcommand{\subsection}[1]{
	\oldsubsection{#1}
	\expandafter\addtostream{CONTENT}{\noindent\hspace{6ex}\unexpanded{\stepcounter{SUB}\theSUB ~ #1}\par}
}


%\renewcommand{\section}[1]{\MMM{#1}}

%\edef\subsection#1
{
	%\noexpand\subsection{#1}
	%
}


\newfontfamily\dnifamily[Scale = 0.795]{DniFont.TTF}

\newcommand{\dni}[1]{%
	{\dnifamily%
		\ifthenelse{#1=0}{0}{}%
		\ifthenelse{#1=1}{1}{}%
		\ifthenelse{#1=2}{2}{}%
		\ifthenelse{#1=3}{3}{}%
		\ifthenelse{#1=4}{4}{}%
		\ifthenelse{#1=5}{5}{}%
		\ifthenelse{#1=6}{6}{}%
		\ifthenelse{#1=7}{7}{}%
		\ifthenelse{#1=8}{8}{}%
		\ifthenelse{#1=9}{9}{}%
		\ifthenelse{#1=10}{)}{}%
		\ifthenelse{#1=11}{!}{}%
		\ifthenelse{#1=12}{@}{}%
		\ifthenelse{#1=13}{\#}{}%
		\ifthenelse{#1=14}{\$}{}%
		\ifthenelse{#1=15}{\%}{}%
		\ifthenelse{#1=16}{\^{}}{}%
		\ifthenelse{#1=17}{\&}{}%
		\ifthenelse{#1=18}{*}{}%
		\ifthenelse{#1=19}{(}{}%
		\ifthenelse{#1=20}{[}{}%
		\ifthenelse{#1=21}{]}{}%
		\ifthenelse{#1=22}{\textbackslash{}}{}%
		\ifthenelse{#1=23}{\{}{}%
		\ifthenelse{#1=24}{\}}{}%
		\ifthenelse{#1=25}{|}{}}%
}%

\newcommand{\toDni}[1]{%	
	\ifthenelse{#1=0}{}{%
		 \ifthenelse{#1=25}{%
		 	\expandafter\dni{#1}}{%
		 	\expandafter\toDni{\fpeval{floor(#1/25)}}%
		 \expandafter\dni{\fpeval{(#1/25 - floor(#1/25))/0.04}}}}%
}%



\newcommand{\Strut}{{\Large\strut}}

\newcommand\scb[1]{\left( #1 \right)}

\newcommand{\LINK}[2]{%
	\qrcode[height=1cm]{#1}\  \HREF{#1}{\parbox{0.8\textwidth}{#2}} \\[0.5em]
}

\NewDocumentCommand{\lecdni}{}{\toDni{\LectionNumber}}
\author{Гаврилов Андрей Геннадьевич}
\newcommand{\regals}{кандидат технических наук, доцент}
\institute{Кафедра Информационных технологий и вычислительных систем \\МГТУ~<<СТАНКИН>>}
\lecture{История компьютерной графики}{kghistory}\subtitle{Компьютерная графика}


\makeatletter
\newcommand*{\overlaynumber}{\number\beamer@slideinframe}
\makeatother



\usepackage{cprotect}

\newcommand{\QRFRAME}{%
    \begin{frame}[plain, noframenumbering]    	
	
	\centering
	Трансляция презентации (во время очных лекций)    
	
	~
	
	{\Large \ttfamily  https://clck.ru/3D3Efj  }
	
	~
	
	\tikz\node[inner sep=0pt,rounded corners=5mm, clip]{\qrcode[height=0.45\textwidth]{\LECADDR}}; 
	
	~	
	{\small
	При просмотре презентации в PDF для отображения анимаций на слайдах необходимо использовать Acrobat Reader, KDE Okular, PDF-XChange, Foxit Reader, браузер Firefox. Для браузеров на движке Chrome (Edge, Яндекс, Opera,~\dots) необходимо использовать \HREF{https://chromewebstore.google.com/detail/pdf-viewer/oemmndcbldboiebfnladdacbdfmadadm?hl=ru&utm_source=ext_sidebar}{PDF.js} c опцией <<Enable active content (JavaScript) in PDFs>>. }
	
	\end{frame}%
}

\newcommand{\IG}[2][1]{\includegraphics[width=#1\textwidth]{#2}}



\graphicspath{{Images/}{Images/\jobname/}}

\date{\today}



\renewcommand{\LectionNumber}{11}
\renewcommand{\LectionTheme}{Наложение специальных текстур}
\title{Лекция \lecdni \\ \LectionTheme}
\subtitle{Компьютерная графика}



%\usepackage{standalone}

\setbeamersize
{
	text margin left=0.5cm,
	text margin right=0.5cm
}

\usepackage{comment}


%	\transduration{2}
%   \transfade

 \begin{document}
 		 
	\makeatletter
\defbeamertemplate*{title page}{my theme}
{
	
	\hfill
	
	\begin{beamercolorbox}[wd=.9\paperwidth,center,]{title}%
		
	\end{beamercolorbox}%	
	
	\vbox to 1em {}
	
	\begin{beamercolorbox}[wd=.9\paperwidth,center,]{title}%
		\usebeamerfont{subtitle}%
		\hfill
		
		\insertsubtitle
		
		\usebeamerfont{title}%
		\inserttitle{} \\[0.5em]
		
		
		
	\end{beamercolorbox}%	
	\hfill\hfill
	
	\begin{beamercolorbox}[wd=.9\paperwidth,center,]{}%
		\usebeamerfont{author}%
		\hfill \\[0.5em]
		\insertauthor{} \\[0.5em]
		\regals
		    
		\vbox to 1em{}
		\usebeamerfont{institute}%
		\insertinstitute {}
		
		\vbox to 1em{}			
		{\; }\insertshortdate{}
		
	\end{beamercolorbox}%	
	\hfill\hfill
	
	\vbox to 5em{}
	
	
}
\defbeamertemplate*{footline}{my theme}{
	\leavevmode%
	\hbox{%
		\begin{beamercolorbox}[wd=.25\paperwidth,ht=3.25ex,dp=0ex,center,sep=1pt]{author in head/foot}%
			\usebeamerfont{author in head/foot}%
			\insertauthor 
			\beamer@ifempty{\insertshortinstitute}{}
		\end{beamercolorbox}%
		\begin{beamercolorbox}[wd=.65\paperwidth,ht=3.25ex,dp=0ex,center,sep=1pt]{title in head/foot}%
			\usebeamerfont{title in head/foot}\insertshortinstitute
		\end{beamercolorbox}%
		\begin{beamercolorbox}[wd=.1\paperwidth,ht=3.25ex,dp=0ex,center, sep=0.5pt]{date in head/foot}%
			\usebeamerfont{date in head/foot}
			\footnotesize \expandafter\toDni{\insertframenumber} / \expandafter\toDni{\inserttotalframenumber}
	\end{beamercolorbox}}%
}



\makeatother






%float page top aligment
\makeatletter
\setlength{\@fptop}{0pt}
\setlength{\@fpbot}{0pt plus 1fil}
\makeatother

\newcommand \abs[1] {\left| #1 \right|}

\everymath{\displaystyle}

    
    \QRFRAME	
	

	\frame{\maketitle}

	
	\begin{frame}{План лекции}
		%\tableofcontents
	\end{frame}
	

\begin{frame}{Displacement mapping}
	
	\TC{0.5}
	{
		\IG{Displacement_Mapping.jpg}
	}{
		\IG{dm_3_small.jpg}
		
		~
		
		~
		
		\IG{dm_small.jpg}
	}
	
	
	
\end{frame}
	
\begin{frame}{Bump mapping}
	
	\TC{0.5}
	{
		\animategraphics[autoplay,loop, nomouse, width=\textwidth]{20}{Images/\jobname/bump_anim/bump}{0001}{0150}
	}{
		\centering\IG{stonebrick.png}
	}
\end{frame}	

\begin{frame}{Карта высот}
	
	\centering\IG[0.9]{StandardShaderHeightmapTexture.png}
	
\end{frame}

\begin{frame}{Рельеф то не настоящий!}
	
	\centering\animategraphics[autoplay,loop, nomouse, width=0.6\textwidth]{20}{Images/\jobname/bump2/bump2}{0001}{0150}
	
\end{frame}


\begin{frame}{Где "живет" нормаль?}
	
	\TC{0.5}
	{
		\IG{1493389010153817205.jpg}
	}{
		Карта высот --- $f(x,y,z)=\mathrm{const}$.
		$\vv n_f = \nabla f = \left( \dfrac{\partial f}{\partial x},\dfrac{\partial f}{\partial y},\dfrac{\partial f}{\partial z} \right)$
		
		$f(x,y,z) = h(x,y) -z=0$
		
		~
		
		$\vv n(x,y) = \left( \dfrac{\partial f}{\partial x},\dfrac{\partial f}{\partial y},-1 \right)$
	}
	$\vv n(x,y) = \left(\dfrac{h(x+1,y)-h(x-1,y) }{2},\dfrac{h(x,y+1)-h(x,y-1) }{2}, -1  \right) $
\end{frame}
	
\begin{frame}{Касательное пространство (tangent space)}
	
	\TC{0.4}
	{
		\centering
		
		\only<1>{\includegraphics[page=1]{TNB.pdf}}%
		\only<2->{\includegraphics[page=2]{TNB.pdf}}%
		
		
		\includegraphics[page=3]{TNB.pdf}
		
	}{
		
		$\begin{array}{l}
			\vv e_1 = P_2-P_1 \\
			\vv e_2 = P_3-P_2 \\
		\end{array}$
		\pause $\Rightarrow$
		$\begin{array}{l}
			\vv e_1 = \Delta u_1\vv T+ \Delta v_1\vv B \\
			\vv  e_2 = \Delta u_2\vv T+ \Delta v_2\vv B\\
		\end{array}$
		
		\pause
		
		или так:
		
		$\begin{pmatrix}
			\vv e_1 \\
			\vv e_2
		\end{pmatrix}
		=
		\begin{pmatrix}
			\Delta u_1 & \Delta v_1\\
			\Delta u_2 & \Delta v_2
		\end{pmatrix}
		\begin{pmatrix}
			\vv T \\
			\vv B
		\end{pmatrix}
		$
		
		~ \pause
		
		~
		
		$	
		\begin{pmatrix}
			\vv T \\
			\vv B
		\end{pmatrix}
		=
		\begin{pmatrix}
			\Delta u_1 & \Delta v_1\\
			\Delta u_2 & \Delta v_2
		\end{pmatrix} ^ {-1}
		\begin{pmatrix}
			\vv e_1 \\
			\vv e_2
		\end{pmatrix}
		$
		
		$	
		\begin{pmatrix}
			\vv T \\
			\vv B
		\end{pmatrix}
		=
		\dfrac{
			\begin{pmatrix}
				\Delta v_1 & -\Delta v_1\\
				-\Delta u_2 & \Delta u_2
			\end{pmatrix}
		}
		{	
			\Delta u_1 \Delta v_2 - \Delta v_1 \Delta u_2
		}	
		\begin{pmatrix}
			\vv e_1 \\
			\vv e_2
		\end{pmatrix}
		$
		
		~
		
		\pause$
		\mathbf M_{TBN}=
		\begin{pmatrix}
			\vv T\\
			\vv B\\
			\vv N\\
		\end{pmatrix}
		=
		\begin{pmatrix}
			T_x & T_y & T_z\\
			B_x & B_y & B_z\\
			N_x & N_y & N_z\\
		\end{pmatrix}
		$	
		
		$\mathbf A^\text{мир} = \mathbf M_{TBN} \mathbf A^\text{кас}, \ $
		$\mathbf A^\text{кас} = {\mathbf M _{TBN}}^{-1} \mathbf A^\text{мир}$
		
	}
	
	
	
\end{frame}	

\begin{frame}{Normal mapping}
		
		\centering
		
		\IG[1]{0cppvtgpjw48at0r2crvdy3omfg.png}
		
		\IG[1]{kay_684wpm4mpkko92gl1nemjh4.png}

\end{frame}
	
\begin{frame}{Карта нормалей}
	
	\centering
	\TC{0.5}
	{
		\IG{8620ac5aaca576891aaa13188c1d3a75.jpg}	
	}{
	    \IG{c4ef222ce73d3cd29b0b055cec587c6b.jpg}	
	}

	
\end{frame}
	
\begin{frame}{Виды карт нормалей}
	
	\IG{vtf_botw_8tazlodtepbgp0zyvk.png}
	
\end{frame}

\begin{frame}{Результат наложения карты нормалей}
	\TC{0.5}
	{
		\animategraphics[autoplay,loop, nomouse, width=\textwidth]{25}{Images/\jobname/nm_anim/}{0001}{0150}
	}{
		\IG{nm2.png}
	}
	
\end{frame}



\begin{frame}{Создание карт нормалей}	
	
	\TC{0.5}
	{
		\IG{e4ebbbb943f8a3ad71016ee396bc55bb.jpg}
		
		\IG{2565084f33aaafaedfb752e7b5c01b89}
		
	}{
		\IG{72c48753443123074b3b0170ee04e7e8}
	}
	
\end{frame}

\begin{frame}{Lowpoly и highpoly модели}
	
	\IG[0.95]{5b3d7c86fc1fe61040b1b7b0292b24572ecebc6f-1.jpg}
	
	
\end{frame}
	

\begin{frame}{Paralax mapping}
	
	\TC{0.4}
	{
		\IG{iinoiejzmk64kssow31hqnm4bd4}
	}{
		\IG{uubupwg7tnqbjpv3rhf2p5xjw4y}
	}
	
	\HREF{https://cpetry.github.io/NormalMap-Online/}{Демонстрация}
	
	
\end{frame}

\begin{frame}{Основная идея}
	
	\IG {rwabtldbgzyc-rsarcgipyteab0.png}
	
	\IG [0.48] {p7kazlkvcdvy53_r05zpmwedros.png} \IG[0.48]{mbbqtsva5kzyzpj8ettleqrdxr4.png}
	
\end{frame}

\begin{frame}{На основе карты грубин}
	
	\IG{oqzry5trs2lyfbwhe6puu0u04y8.png}
	
	\begin{tikzpicture}[overlay, remember picture]
		\draw (current page.center) node[xshift=-4cm, yshift=1.3cm]  {\includegraphics[width=3cm]{bricks2_disp.jpg}};
		
	\end{tikzpicture}
	
\end{frame}

\begin{frame}{Что для всего этого нужно}
	
	\TC{0.5}
	{
		\IG{uubupwg7tnqbjpv3rhf2p5xjw4y}
		
	}{
		\centering
		\IG[0.45]{bricks2.jpg}
		\IG[0.45]{bricks2_disp.jpg}
		\IG[0.45]{bricks2_normal.jpg}
	}
	
\end{frame}



\begin{frame}{Трассировка луча}
	
	\IG{Screenshot 2024-10-02 212545.png}
	
\end{frame}

\begin{frame}{Сравнение технологий}
	
	\IG{StandardShaderParallaxMap}
	
\end{frame}



 
\begin{comment}
	
\end{comment}
 

\end{document}