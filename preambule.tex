


\mode<article>{
	
	\usepackage{hyperref}
	
}
\mode<presentation>{
	
	\usetheme{Antibes}
	\usefonttheme{professionalfonts} 
	\usefonttheme{serif} % default family is serif
	
	%\usecolortheme{spruce} %зеленая, плохой цвет в заголовках 
	%\usecolortheme{albatross} %синяя, пхоло виден черный цвет
	
}

\newcommand{\MP}[1]{\mode<presentation>{#1} }
\newcommand{\MA}[1]{\mode<article>{#1} }

\newcommand{\ABS}[1]{\left| #1 \right|}
%\newcommand{\ABS}[1]{\mid #1 \mid}

\newcommand{\HREF}[2]{{\color{blue}\underline{\href{#1}{#2}}}}

\setbeamertemplate{caption}[numbered]


%\usepackage[T2A]{fontenc}
%\usepackage[utf8]{inputenc}
%\usepackage[russian]{babel}
%\usepackage{amsmath} %математические формулы



\usepackage{ifthen}

\usepackage{tikz}
\usetikzlibrary{arrows.meta}
\usetikzlibrary{calc}
\usetikzlibrary{decorations}
\usetikzlibrary{decorations.pathreplacing}
\newcommand{\rememb}[1]{\tikz[remember picture,baseline=-0.5ex]{\draw node[inner sep=0pt, outer sep=0pt] (#1){\strut};}}



\usepackage{fp}
\usepackage{tikz-3dplot}
\usepackage{environ}
\usepackage{animate}





\usepackage{xcolor}
%\usepackage[left=20mm,right=20mm,top=20mm,bottom=20mm,a4paper]{geometry} %поля

\usepackage{amsmath} %математические формулы
%\usepackage{amsfonts} %математические шрифты


\usepackage[e]{esvect}  %Красивая стрелочка вектора
%\let\oldvv\vv
\newcommand{\VV}[1]{\vv{#1\mathstrut}}



\usepackage{graphicx} %работа с каритнками


%\usepackage{multimedia}
%\usepackage{movie15}

%Для XeLatex/+
\usepackage{polyglossia}
\setdefaultlanguage{russian}
\setotherlanguage{english}
%\setkeys{russian}{babelshorthands=true} 


\usepackage{fontspec}

\setmainfont{Times New Roman} [Script=Cyrillic, Mapping=tex-text,]
\setsansfont{Arial} [Script=Cyrillic, Mapping=tex-text,]
%\setmonofont{Courier New} [Script=Cyrillic, Mapping=tex-text,]
\newfontfamily{\cyrillicfonttt}{Courier New}


%\usepackage{unicode-math}
%\setmathfont{TeX Gyre Termes Math}

%\setmainfont{CMU Serif}[Script=Cyrillic, Mapping=tex-text,]
%\setsansfont{CMU Sans Serif}[Script=Cyrillic, Mapping=tex-text,]
%\setmonofont{CMU Typewriter Text}[Script=Cyrillic, Mapping=tex-text,]


%-----------------


%\usepackage{caption}
%\DeclareCaptionLabelSeparator{dot}{~---~}            %Разделитель номер рисунка
%\captionsetup[figure]{justification=centering,labelsep=dot, format=plain}                        %Подпись рис. центр
%\captionsetup[table]{justification=raggedleft,labelsep=dot, format=plain, singlelinecheck=false} %Подпись табл. слева
%\captionsetup[lstlisting]{justification=raggedleft,labelsep=dot, format=plain, singlelinecheck=false}                     %Подпись рис. центр

\usepackage{indentfirst} %отступ первой строки


\usepackage[svgnames]{xcolor}


\usepackage{hyperref}

%\usepackage{showframe}


%\usepackage{tikz}

%\usepackage[hidelinks]{hyperref}%ссылки внутри документа \ref


\setlength\abovecaptionskip{-2pt}
%\setlength\belowcaptionskip{-14pt}

\setbeamerfont{caption}{size=\scriptsize}


\def\sectionname{Раздел}
\def\subsectionname{Подраздел}


\newcommand{\TC}[3]
{
	
	
	\begin{columns}
		\begin{column}{#1\textwidth}
			#2
		\end{column}
		\begin{column}{\fpeval{1-#1}\textwidth}
			#3
		\end{column}
	\end{columns}
}

\newcommand{\TCT}[3]
{
	
	\begin{columns}[T]
		\begin{column}{#1\textwidth}
			#2
		\end{column}
		\begin{column}{\fpeval{1-#1}\textwidth}
			#3
		\end{column}
	\end{columns}
}


\newcommand{\FRAME}[2]{
	\begin{frame}
		\frametitle{#1}
		#2
	\end{frame}
}

\newcommand{\FIG}[3]
{
	\begin{figure}
		\centering
		\includegraphics[width=#3]{#1}
		\caption{#2}
	\end{figure}
}

\newcommand{\vect}[1]{\overrightarrow{#1}}


\usepackage{qrcode}

\newcommand{\LECADDR}{https://clck.ru/3D3Efj}


\usepackage{newfile}

\edef\LectionNumber{0}
\edef\LectionTheme{0}

\let\oldsection\section
\let\oldsubsection\subsection


\AtBeginDocument
{
	\newoutputstream{CONTENT}
	\openoutputfile{\jobname.gvr}{CONTENT}
	
	%\expandafter\addtostream{CONTENT}{\noindent\textbf{\Large Лекция \LectionNumber~---~\LectionTheme}\unexpanded{\setcounter{SEC}{0}}\par}
	\addtostream{CONTENT}{\protect\noindent%
						  	\protect\setcounter{SEC}{0}%
							{%
							\protect\large% 
							\protect\bfseries{}%
							Лекция \LectionNumber~---~\LectionTheme% 
							}% 
							\protect\par}
}

\renewcommand{\section}[1]{
	\oldsection{#1}
	\addtostream{CONTENT}{%
			\protect\noindent%
			\protect\stepcounter{SEC}%
			\protect\setcounter{SUB}{0}%
			\protect\hspace{3ex}%
			\protect\theSEC ~ #1 \protect\par%
			}
}

\renewcommand{\subsection}[1]{
	\oldsubsection{#1}
	\addtostream{CONTENT}{%
			\protect\noindent%
			\protect\stepcounter{SUB}%
			\protect\hspace{6ex}%
			\protect\theSEC{}.\protect\theSUB ~ #1 \protect\par%
			}
}


%\renewcommand{\section}[1]{\MMM{#1}}

%\edef\subsection#1
{
	%\noexpand\subsection{#1}
	%
}


\newfontfamily\dnifamily[Scale = 0.795]{DniFont.TTF}

\newcommand{\dni}[1]{%
	{\dnifamily%
		\ifthenelse{#1=0}{0}{}%
		\ifthenelse{#1=1}{1}{}%
		\ifthenelse{#1=2}{2}{}%
		\ifthenelse{#1=3}{3}{}%
		\ifthenelse{#1=4}{4}{}%
		\ifthenelse{#1=5}{5}{}%
		\ifthenelse{#1=6}{6}{}%
		\ifthenelse{#1=7}{7}{}%
		\ifthenelse{#1=8}{8}{}%
		\ifthenelse{#1=9}{9}{}%
		\ifthenelse{#1=10}{)}{}%
		\ifthenelse{#1=11}{!}{}%
		\ifthenelse{#1=12}{@}{}%
		\ifthenelse{#1=13}{\#}{}%
		\ifthenelse{#1=14}{\$}{}%
		\ifthenelse{#1=15}{\%}{}%
		\ifthenelse{#1=16}{\^{}}{}%
		\ifthenelse{#1=17}{\&}{}%
		\ifthenelse{#1=18}{*}{}%
		\ifthenelse{#1=19}{(}{}%
		\ifthenelse{#1=20}{[}{}%
		\ifthenelse{#1=21}{]}{}%
		\ifthenelse{#1=22}{\textbackslash{}}{}%
		\ifthenelse{#1=23}{\{}{}%
		\ifthenelse{#1=24}{\}}{}%
		\ifthenelse{#1=25}{|}{}}%
}%

\newcommand{\toDni}[1]{%	
	\ifthenelse{#1=0}{}{%
		 \ifthenelse{#1=25}{%
		 	\expandafter\dni{#1}}{%
		 	\expandafter\toDni{\fpeval{floor(#1/25)}}%
		 \expandafter\dni{\fpeval{(#1/25 - floor(#1/25))/0.04}}}}%
}%



\newcommand{\Strut}{{\Large\strut}}

\newcommand\scb[1]{\left( #1 \right)}

\newcommand{\LINK}[2]{%
	\qrcode[height=1cm]{#1}\  \HREF{#1}{\parbox{0.8\textwidth}{#2}} \\[0.5em]
}

\NewDocumentCommand{\lecdni}{}{\toDni{\LectionNumber}}
\author{Гаврилов Андрей Геннадьевич}
\newcommand{\regals}{кандидат технических наук, доцент}
\institute{Кафедра Информационных технологий и вычислительных систем \\МГТУ~<<СТАНКИН>>}
\lecture{История компьютерной графики}{kghistory}\subtitle{Компьютерная графика}


\makeatletter
\newcommand*{\overlaynumber}{\number\beamer@slideinframe}
\makeatother



\usepackage{cprotect}

\newcommand{\QRFRAME}{%
    \begin{frame}[plain, noframenumbering]    	
	
	\centering
	Трансляция презентации (во время очных лекций)    
	
	~
	
	{\Large \ttfamily  https://clck.ru/3D3Efj  }
	
	~
	
	\tikz\node[inner sep=0pt,rounded corners=5mm, clip]{\qrcode[height=0.45\textwidth]{\LECADDR}}; 
	
	~	
	{\small
	При просмотре презентации в PDF для отображения анимаций на слайдах необходимо использовать Acrobat Reader, KDE Okular, PDF-XChange, Foxit Reader, браузер Firefox. Для браузеров на движке Chrome (Edge, Яндекс, Opera,~\dots) необходимо использовать \HREF{https://chromewebstore.google.com/detail/pdf-viewer/oemmndcbldboiebfnladdacbdfmadadm?hl=ru&utm_source=ext_sidebar}{PDF.js} c опцией <<Enable active content (JavaScript) in PDFs>>. }
	
	\end{frame}%
}

\newcommand{\IG}[2][1]{\includegraphics[width=#1\textwidth]{#2}}

