\documentclass[10pt]{beamer}

%\documentclass[10pt, handout]{beamer}
\setbeameroption{show notes}

%\documentclass[10pt, a4paper]{article}
%\usepackage{beamerarticle}




\mode<article>{
	
	\usepackage{hyperref}
	
}
\mode<presentation>{
	
	\usetheme{Antibes}
	\usefonttheme{professionalfonts} 
	\usefonttheme{serif} % default family is serif
	
	%\usecolortheme{spruce} %зеленая, плохой цвет в заголовках 
	%\usecolortheme{albatross} %синяя, пхоло виден черный цвет
	
}

\newcommand{\MP}[1]{\mode<presentation>{#1} }
\newcommand{\MA}[1]{\mode<article>{#1} }

\newcommand{\ABS}[1]{\left| #1 \right|}
%\newcommand{\ABS}[1]{\mid #1 \mid}

\newcommand{\HREF}[2]{{\color{blue}\underline{\href{#1}{#2}}}}

\setbeamertemplate{caption}[numbered]


%\usepackage[T2A]{fontenc}
%\usepackage[utf8]{inputenc}
%\usepackage[russian]{babel}
%\usepackage{amsmath} %математические формулы



\usepackage{ifthen}

\usepackage{tikz}
\usetikzlibrary{arrows.meta}
\usetikzlibrary{calc}
\usetikzlibrary{decorations}
\usetikzlibrary{decorations.pathreplacing}
\newcommand{\rememb}[1]{\tikz[remember picture,baseline=-0.5ex]{\draw node[inner sep=0pt, outer sep=0pt] (#1){\strut};}}



\usepackage{fp}
\usepackage{tikz-3dplot}
\usepackage{environ}
\usepackage{animate}





\usepackage{xcolor}
%\usepackage[left=20mm,right=20mm,top=20mm,bottom=20mm,a4paper]{geometry} %поля

\usepackage{amsmath} %математические формулы
%\usepackage{amsfonts} %математические шрифты


\usepackage[e]{esvect}  %Красивая стрелочка вектора
%\let\oldvv\vv
\newcommand{\VV}[1]{\vv{#1\mathstrut}}



\usepackage{graphicx} %работа с каритнками


%\usepackage{multimedia}
%\usepackage{movie15}

%Для XeLatex/+
\usepackage{polyglossia}
\setdefaultlanguage{russian}
\setotherlanguage{english}
%\setkeys{russian}{babelshorthands=true} 


\usepackage{fontspec}

\setmainfont{Times New Roman} [Script=Cyrillic, Mapping=tex-text,]
\setsansfont{Arial} [Script=Cyrillic, Mapping=tex-text,]
%\setmonofont{Courier New} [Script=Cyrillic, Mapping=tex-text,]
\newfontfamily{\cyrillicfonttt}{Courier New}


%\usepackage{unicode-math}
%\setmathfont{TeX Gyre Termes Math}

%\setmainfont{CMU Serif}[Script=Cyrillic, Mapping=tex-text,]
%\setsansfont{CMU Sans Serif}[Script=Cyrillic, Mapping=tex-text,]
%\setmonofont{CMU Typewriter Text}[Script=Cyrillic, Mapping=tex-text,]


%-----------------


%\usepackage{caption}
%\DeclareCaptionLabelSeparator{dot}{~---~}            %Разделитель номер рисунка
%\captionsetup[figure]{justification=centering,labelsep=dot, format=plain}                        %Подпись рис. центр
%\captionsetup[table]{justification=raggedleft,labelsep=dot, format=plain, singlelinecheck=false} %Подпись табл. слева
%\captionsetup[lstlisting]{justification=raggedleft,labelsep=dot, format=plain, singlelinecheck=false}                     %Подпись рис. центр

\usepackage{indentfirst} %отступ первой строки


\usepackage[svgnames]{xcolor}


\usepackage{hyperref}

%\usepackage{showframe}


%\usepackage{tikz}

%\usepackage[hidelinks]{hyperref}%ссылки внутри документа \ref


\setlength\abovecaptionskip{-2pt}
%\setlength\belowcaptionskip{-14pt}

\setbeamerfont{caption}{size=\scriptsize}


\def\sectionname{Раздел}
\def\subsectionname{Подраздел}


\newcommand{\TC}[3]
{
	
	
	\begin{columns}
		\begin{column}{#1\textwidth}
			#2
		\end{column}
		\begin{column}{\fpeval{1-#1}\textwidth}
			#3
		\end{column}
	\end{columns}
}

\newcommand{\TCT}[3]
{
	
	\begin{columns}[T]
		\begin{column}{#1\textwidth}
			#2
		\end{column}
		\begin{column}{\fpeval{1-#1}\textwidth}
			#3
		\end{column}
	\end{columns}
}


\newcommand{\FRAME}[2]{
	\begin{frame}
		\frametitle{#1}
		#2
	\end{frame}
}

\newcommand{\FIG}[3]
{
	\begin{figure}
		\centering
		\includegraphics[width=#3]{#1}
		\caption{#2}
	\end{figure}
}

\newcommand{\vect}[1]{\overrightarrow{#1}}


\usepackage{qrcode}

\newcommand{\LECADDR}{https://clck.ru/3D3Efj}


\usepackage{newfile}

\edef\LectionNumber{0}
\edef\LectionTheme{0}

\let\oldsection\section
\let\oldsubsection\subsection


\AtBeginDocument
{
	\newoutputstream{CONTENT}
	\openoutputfile{\LectionNumber .gvr}{CONTENT}
	
	\expandafter\addtostream{CONTENT}{\noindent\textbf{\Large Лекция \LectionNumber~---~\LectionTheme}\unexpanded{\setcounter{SEC}{0}}\par}
}

\renewcommand{\section}[1]{
	\oldsection{#1}
	\expandafter\addtostream{CONTENT}{\noindent\hspace{2ex}\unexpanded{\hbox{\large\stepcounter{SEC}\theSEC ~ #1}}\par}
}

\renewcommand{\subsection}[1]{
	\oldsubsection{#1}
	\expandafter\addtostream{CONTENT}{\noindent\hspace{6ex}\unexpanded{\stepcounter{SUB}\theSUB ~ #1}\par}
}


%\renewcommand{\section}[1]{\MMM{#1}}

%\edef\subsection#1
{
	%\noexpand\subsection{#1}
	%
}


\newfontfamily\dnifamily[Scale = 0.795]{DniFont.TTF}

\newcommand{\dni}[1]{%
	{\dnifamily%
		\ifthenelse{#1=0}{0}{}%
		\ifthenelse{#1=1}{1}{}%
		\ifthenelse{#1=2}{2}{}%
		\ifthenelse{#1=3}{3}{}%
		\ifthenelse{#1=4}{4}{}%
		\ifthenelse{#1=5}{5}{}%
		\ifthenelse{#1=6}{6}{}%
		\ifthenelse{#1=7}{7}{}%
		\ifthenelse{#1=8}{8}{}%
		\ifthenelse{#1=9}{9}{}%
		\ifthenelse{#1=10}{)}{}%
		\ifthenelse{#1=11}{!}{}%
		\ifthenelse{#1=12}{@}{}%
		\ifthenelse{#1=13}{\#}{}%
		\ifthenelse{#1=14}{\$}{}%
		\ifthenelse{#1=15}{\%}{}%
		\ifthenelse{#1=16}{\^{}}{}%
		\ifthenelse{#1=17}{\&}{}%
		\ifthenelse{#1=18}{*}{}%
		\ifthenelse{#1=19}{(}{}%
		\ifthenelse{#1=20}{[}{}%
		\ifthenelse{#1=21}{]}{}%
		\ifthenelse{#1=22}{\textbackslash{}}{}%
		\ifthenelse{#1=23}{\{}{}%
		\ifthenelse{#1=24}{\}}{}%
		\ifthenelse{#1=25}{|}{}}%
}%

\newcommand{\toDni}[1]{%	
	\ifthenelse{#1=0}{}{%
		 \ifthenelse{#1=25}{%
		 	\expandafter\dni{#1}}{%
		 	\expandafter\toDni{\fpeval{floor(#1/25)}}%
		 \expandafter\dni{\fpeval{(#1/25 - floor(#1/25))/0.04}}}}%
}%



\newcommand{\Strut}{{\Large\strut}}

\newcommand\scb[1]{\left( #1 \right)}

\newcommand{\LINK}[2]{%
	\qrcode[height=1cm]{#1}\  \HREF{#1}{\parbox{0.8\textwidth}{#2}} \\[0.5em]
}

\NewDocumentCommand{\lecdni}{}{\toDni{\LectionNumber}}
\author{Гаврилов Андрей Геннадьевич}
\newcommand{\regals}{кандидат технических наук, доцент}
\institute{Кафедра Информационных технологий и вычислительных систем \\МГТУ~<<СТАНКИН>>}
\lecture{История компьютерной графики}{kghistory}\subtitle{Компьютерная графика}


\makeatletter
\newcommand*{\overlaynumber}{\number\beamer@slideinframe}
\makeatother



\usepackage{cprotect}

\newcommand{\QRFRAME}{%
    \begin{frame}[plain, noframenumbering]    	
	
	\centering
	Трансляция презентации (во время очных лекций)    
	
	~
	
	{\Large \ttfamily  https://clck.ru/3D3Efj  }
	
	~
	
	\tikz\node[inner sep=0pt,rounded corners=5mm, clip]{\qrcode[height=0.45\textwidth]{\LECADDR}}; 
	
	~	
	{\small
	При просмотре презентации в PDF для отображения анимаций на слайдах необходимо использовать Acrobat Reader, KDE Okular, PDF-XChange, Foxit Reader, браузер Firefox. Для браузеров на движке Chrome (Edge, Яндекс, Opera,~\dots) необходимо использовать \HREF{https://chromewebstore.google.com/detail/pdf-viewer/oemmndcbldboiebfnladdacbdfmadadm?hl=ru&utm_source=ext_sidebar}{PDF.js} c опцией <<Enable active content (JavaScript) in PDFs>>. }
	
	\end{frame}%
}

\newcommand{\IG}[2][1]{\includegraphics[width=#1\textwidth]{#2}}




\usepackage{media9}

\graphicspath{{Images/}{Images/\jobname/}}

\date{\today}


\renewcommand{\LectionNumber}{12}
\renewcommand{\LectionTheme}{Представление кривых и поверхностей}
\title{Лекция \lecdni \\ \LectionTheme}
\subtitle{Компьютерная графика}



%\usepackage{standalone}

\setbeamersize
{
	text margin left=0.5cm,
	text margin right=0.5cm
}

\usepackage{comment}


%	\transduration{2}
%   \transfade




\begin{document}
 		 
	\makeatletter
\defbeamertemplate*{title page}{my theme}
{
	
	\hfill
	
	\begin{beamercolorbox}[wd=.9\paperwidth,center,]{title}%
		
	\end{beamercolorbox}%	
	
	\vbox to 1em {}
	
	\begin{beamercolorbox}[wd=.9\paperwidth,center,]{title}%
		\usebeamerfont{subtitle}%
		\hfill
		
		\insertsubtitle
		
		\usebeamerfont{title}%
		\inserttitle{} \\[0.5em]
		
		
		
	\end{beamercolorbox}%	
	\hfill\hfill
	
	\begin{beamercolorbox}[wd=.9\paperwidth,center,]{}%
		\usebeamerfont{author}%
		\hfill \\[0.5em]
		\insertauthor{} \\[0.5em]
		\regals
		    
		\vbox to 1em{}
		\usebeamerfont{institute}%
		\insertinstitute {}
		
		\vbox to 1em{}			
		{\; }\insertshortdate{}
		
	\end{beamercolorbox}%	
	\hfill\hfill
	
	\vbox to 5em{}
	
	
}
\defbeamertemplate*{footline}{my theme}{
	\leavevmode%
	\hbox{%
		\begin{beamercolorbox}[wd=.25\paperwidth,ht=3.25ex,dp=0ex,center,sep=1pt]{author in head/foot}%
			\usebeamerfont{author in head/foot}%
			\insertauthor 
			\beamer@ifempty{\insertshortinstitute}{}
		\end{beamercolorbox}%
		\begin{beamercolorbox}[wd=.65\paperwidth,ht=3.25ex,dp=0ex,center,sep=1pt]{title in head/foot}%
			\usebeamerfont{title in head/foot}\insertshortinstitute
		\end{beamercolorbox}%
		\begin{beamercolorbox}[wd=.1\paperwidth,ht=3.25ex,dp=0ex,center, sep=0.5pt]{date in head/foot}%
			\usebeamerfont{date in head/foot}
			\footnotesize \expandafter\toDni{\insertframenumber} / \expandafter\toDni{\inserttotalframenumber}
	\end{beamercolorbox}}%
}



\makeatother






%float page top aligment
\makeatletter
\setlength{\@fptop}{0pt}
\setlength{\@fpbot}{0pt plus 1fil}
\makeatother



\newcommand \abs[1] {\left| #1 \right|}

\everymath{\displaystyle}

    
\begin{comment}
\end{comment}


    
    
    \QRFRAME	
	

	\frame{\maketitle}

	
	\begin{frame}\frametitle{План лекции}
		\tableofcontents
	\end{frame}
	

\section{Неявное представление кривых}
\frame{\sectionpage}

\begin{frame}\frametitle{Неявное представление}
	
	\TC{0.6}
	{
		\includegraphics{function}
		
	}{
		
		Явное: 	$y=f(x)$
		
		~
		
		{\color{red}$y=x^2$}
		
		{\color{green}$y=\sin x$}
		
		{\color{blue}$y=\dfrac{1}{x}$}
		
		~ \pause
		
		Неявное: {\color{orange}$f(x,y) = 0$}
		
	}
	
\end{frame}

\begin{frame}\frametitle{Неявные уравнения второй степени}
	
	\centering
	$ax^2+2bxy+cy^2+2dx+2ey+f=0$
	\IG[0.6]{Screenshot 2024-10-15 164404.jpg}
	
	
\end{frame}

\begin{frame}\frametitle{Параметрическое представление}
	
	\TC{0.5}
	{
		
		\animategraphics[autoplay,loop, nomouse]{15}{Images/\jobname/line}{0}{100}
		
	}{
		$
		f(t)=
		\begin{cases}
		x=x(t),\\
		y=y(t),	
		\end{cases}
		$
		
		~
		
		$
		\vv P(t)=\left[x(t),y(t)\right].
		$
		
		~
		
		
		Для прямой:
		
		$\vv P(t) = \vv P_1+(\vv P_2-\vv P_1)t$,
		
		$x(t)=x_1+(x_2-x_1)t$,
		
		$y(t)=y_1+(y_2-y_1)t$.
	}
	
\end{frame}

\section{Параметрическое представление}
\frame{\sectionpage}

\begin{frame}\frametitle{Параметрическое представление окружности}
	\TC{0.4}
	{
		\animategraphics[autoplay,loop, nomouse]{15}{Images/\jobname/circle}{0}{99}
		
		$
		f(t)=
		\begin{cases}
			x(t)=\cos(t),\\
			y(t)=\sin(t),
		\end{cases}
		$
		
		$
		t \in [0, 2\pi].
		$
		
	}{
		\animategraphics[autoplay,loop, nomouse]{15}{Images/\jobname/circle}{100}{199}
	}
		
	
\end{frame}

\begin{frame}\frametitle{Аффинные преобразования}
	
	\TC{0.5}
	{
		\animategraphics[autoplay,loop, nomouse]{15}{Images/\jobname/parabola/parabola}{0}{100}	
	}{
		$
		f(t)=
		\begin{cases}
			x(t)=t,\\
			y(t)=t^2+t-0.5.
		\end{cases}
		$
		
		\vspace{2ex}
		
		$
		\mathbf R(\varphi)=
		\begin{pmatrix}
			\cos\varphi & -\sin\varphi \\
			\sin\varphi &  \cos\varphi
		\end{pmatrix}.
		$
		
		\vspace{2ex}	

	}
	\TC{0.3}
	{
		
	}{
	
	$
	\begin{pmatrix}
		\cos\varphi & -\sin\varphi \\
		\sin\varphi &  \cos\varphi
	\end{pmatrix}
	\begin{pmatrix}
		x(t)\\
		y(t)
	\end{pmatrix}
	=
	\begin{pmatrix}
		x(t)\cos\varphi - y(t)\sin\varphi  \\
		x(t)\sin\varphi  + y(t)\cos\varphi
	\end{pmatrix}.
	$
		
	}
	

	
\end{frame}

\begin{frame}\frametitle{Параметрическое представление эллипса}

\TC{0.4}
{
	\includegraphics{elipce/elipce}
}{
	$
	f(t)=
	\begin{cases}
		x(t)=a\cos(t),\\
		y(t)=b\sin(t),
	\end{cases}
	a\neq b.
	$
}	



	
\end{frame}

\section{Кривые Безье}
\frame{\sectionpage}

\begin{frame}\frametitle{Кривые Безье}
	
	\TC{0.5}
	{
		
		\animategraphics[autoplay,loop, nomouse]{15}{Images/\jobname/build_bezier_anim}{0}{99}
		
	}{
		\fbox{$\VV C(t)=\sum_{i=0}^nB_{i,n}(t)\VV P_i, t\in[0,1]$},
		
		для кривой, проходящей через $n+1$ точку. Где
		
		$B_{i,n}(t) = 
		\begin{pmatrix}
			n\\
			i
		\end{pmatrix}
		t^i(1-t)^{n-i},$
		
		$\begin{pmatrix}
			n\\
			i
		\end{pmatrix}
		=
		\dfrac{n!}{i!(n-i)!}.$
	}
	
	
\end{frame}

\begin{frame}\frametitle{Линейные кривые Безье}
	
	\TC{0.5}
	{
		\animategraphics[autoplay,loop, nomouse]{15}{Images/\jobname/build_bezier_anim_n2}{0}{114}
	}{
		\begin{center}
			\includegraphics[page=1]{bezierfunctions}
		\end{center}
		$\VV C(t)=\sum_{i=0}^1B_{i,1}(t)\VV P_i$,
		$\VV C(t) = B_{0,1}(t) \VV P_0+B_{1,1}(t) \VV P_1$,
		
		\fbox{$\VV C(t)= \VV P_0(1-t)+\VV P_1t$}.
	}
	
\end{frame}

\begin{frame}\frametitle{Квадратичные кривые Безье}

	\TC{0.5}
	{
		\animategraphics[autoplay,loop, nomouse]{15}{Images/\jobname/build_bezier_anim_n3}{0}{114}
	}{
		\begin{center}
			\includegraphics[page=2]{bezierfunctions}
		\end{center}
		
		$\VV C(t)=\sum_{i=0}^2B_{i,2}(t)\VV P_i$,
		
		$\VV C(t) = B_{0,2}(t) \VV P_0+B_{1,2}(t) \VV P_1+B_{2,2}(t) \VV P_2$,
		
		\fbox{$\VV C(t)=(1-t^2)\VV P_0+2t(1-t)\VV P_1 + t^2\VV P_2$}.
	}	
	
\end{frame}

\begin{frame}\frametitle{Кубические кривые Безье}
	\TC{0.6}
	{
		\animategraphics[autoplay,loop, nomouse]{15}{Images/\jobname/build_bezier_anim_n4}{0}{114}
	}{
		\begin{center}
			\includegraphics[page=3]{bezierfunctions}
		\end{center}
		
		~
		
		$\VV C(t)=\sum_{i=0}^3B_{i,3}(t)\VV P_i$,

	}
	
	$\VV C(t) = B_{0,3}(t) \VV P_0+B_{1,3}(t) \VV P_2+B_{2,3}(t) \VV P_3+B_{3,3}(t) \VV P_1$,
	
	\fbox{$\VV C(t) = (1-t)^3\VV {P}_0 + 3t(1-t)^2\VV{P}_1 + 3t^2(1-t)\VV{P}_2 + t^3\VV{P}_3$}.
	
\end{frame}

\begin{frame}\frametitle{Куби-кубическая или ква-ква-квадратичная кривая}
	
	\centering
	\animategraphics[autoplay,loop, nomouse]{15}{Images/\jobname/build_bezier_anim_n6}{0}{114}
	
\end{frame}


\begin{frame}\frametitle{Рекурсивное определение кривых Безье (алгоритм де Кастельжо)}
	
{
			\centering
			\animategraphics[autoplay,loop, nomouse,scale=0.9, palindrome]{15}{Images/\jobname/recursive_bezier_anim_n3}{0}{198}%
			\animategraphics[autoplay,loop, nomouse,scale=0.9,palindrome]{15}{Images/\jobname/recursive_bezier_anim_n4}{0}{198}
}

	\vspace{-1.5ex}
	$\VV C(t)=\sum_{i=0}^nB_{i,n}(t)\VV P_i \quad \Rightarrow \quad \VV C(t) = \VV\beta_0^{(n)}(t)$,
	
	$\VV\beta_i^{(j)} = \VV\beta_i^{(j-1)}(t) [1-t] + \VV\beta_{i+1}^{(j-1)}(t) t\;, \quad j= n,\ldots,1\;, \ i = 0,\ldots,n-j\;$,
	
	$\VV\beta_i^{(0)}(t)=\VV P_i, \forall t$.
	
\end{frame}


\begin{frame}\frametitle{Весовые функции кривых Безье}
	
	\includegraphics[page=1]{bezierfunctions}
	\includegraphics[page=2]{bezierfunctions}
	\includegraphics[page=3]{bezierfunctions}
	\includegraphics[page=4]{bezierfunctions}
	\includegraphics[page=5]{bezierfunctions}
	\includegraphics[page=6]{bezierfunctions}
	
\end{frame}

\begin{frame}\frametitle{Движение начальной точки}
	
	%C:/Users/gavre/OneDrive/Преподство/Материалы/графика/НоваяПрограмма/Лекции/Images/L12/build_bezier_p0_anim_n6.tex
		\centering
	\animategraphics[autoplay,loop, nomouse, palindrome]{15}{Images/\jobname/build_bezier_p0_anim_n6}{0}{99}
	
\end{frame}

\begin{frame}\frametitle{Движение промежуточной точки}
	
	%C:/Users/gavre/OneDrive/Преподство/Материалы/графика/НоваяПрограмма/Лекции/Images/L12/build_bezier_p0_anim_n6.tex
	\centering
	\animategraphics[autoplay,loop, nomouse, palindrome]{15}{Images/\jobname/build_bezier_p3_anim_n6}{0}{99}
	
\end{frame}

\begin{frame}\frametitle{Безье и матрицы}
	
	$\VV C(t)=\sum_{i=0}^3 B_{i,3}(t) \VV P_i , t\in[0,1]$,
	
	$\VV C(t)=\VV P_0B_{0,3}+\VV P_BB_{1,3}+\VV P_2B_{2,3}+\VV P_3B_{3,3}$=
	
	\hfill$=(1-t)^3\VV P_0+3t(1-t)^2\VV P_1+3t^2(1-t)\VV P_2+t^3\VV P_3$
	
	\pause
	$C(t)=
	\begin{bmatrix}
		(1-t) & 3t(1-t)^2 & 3t^2(1-t) & t^3
	\end{bmatrix}
	\begin{bmatrix}
		P_0\\
		P_1\\
		P_2\\
		P_3
	\end{bmatrix}$,
	$
	C(t)=
	\begin{bmatrix}
		t^3 & t^2 & t & 1
	\end{bmatrix}
	\begin{bmatrix}
		-1& 3& -3&1  \\
		3& -6& 3& 0  \\
		-3& 3&0 &0  \\
		1&0 &0 &0  \\
	\end{bmatrix}
	\begin{bmatrix}
		P_0\\
		P_1\\
		P_2\\
		P_3
	\end{bmatrix}
	=\mathbf{TNG},
	$
	
	\pause
	$
	\overset{n=4}{C(t)}=
	\begin{bmatrix}
		t^4 & t^3 & t^2 & t & 1
	\end{bmatrix}
	$
	$\footnotesize 
	\begin{bmatrix}
		1& -4& 6& -4 &1  \\
		-4& 12& -12& 4 &0 \\
		6& -12& 6 &0 & 0 \\
		-4& 4 &0 &0& 0  \\
		1&0&0&0&0
	\end{bmatrix}
	\begin{bmatrix}
		B_0\\
		B_1\\
		B_2\\
		B_3\\
		B_4
	\end{bmatrix}.
	$
	
	
	
\end{frame}

\begin{frame}\frametitle{Матрицы в общем виде}
	
	
	$\VV C(t)=\mathbf{TNG}$
	
	$\mathbf T = \begin{bmatrix}
		t^n & t^{n-1} & \dots & t & 1
	\end{bmatrix}$
	
	$\mathbf G = \begin{bmatrix}
		P_0 & P_{1} & \dots & P_n
	\end{bmatrix}^T$
	
	\pause
	
	$\mathbf N =
	\begin{bmatrix}
		\left(\begin{smallmatrix}
		n \\ 0
		\end{smallmatrix}\right)
		\left(\begin{smallmatrix}
			n \\ n
		\end{smallmatrix}\right)
		(-1)^n
		&
		\left(\begin{smallmatrix}
			n \\ 1
		\end{smallmatrix}\right)
		\left(\begin{smallmatrix}
			n-1 \\ n-1
		\end{smallmatrix}\right)
		(-1)^{n-1}
		&
		\dots
		&
		\left(\begin{smallmatrix}
			n \\ n
		\end{smallmatrix}\right)
		\left(\begin{smallmatrix}
			n-n \\ n-n
		\end{smallmatrix}\right)
		(-1)^0
		\\
		\left(\begin{smallmatrix}
			n \\ 0
		\end{smallmatrix}\right)
		\left(\begin{smallmatrix}
			n \\ n-1
		\end{smallmatrix}\right)
		(-1)^{n-1}
		&
		\left(\begin{smallmatrix}
			n \\ 1
		\end{smallmatrix}\right)
		\left(\begin{smallmatrix}
			n-1 \\ n-2
		\end{smallmatrix}\right)
		(-1)^{n-2}
		&
		\dots
		&
		0 
		\\
		\vdots & \vdots & \ddots & \vdots \\
		\left(\begin{smallmatrix}
			n \\ 0
		\end{smallmatrix}\right)
		\left(\begin{smallmatrix}
			n \\ 1
		\end{smallmatrix}\right)
		(-1)^{1}
		&
		\left(\begin{smallmatrix}
			n \\ 1
		\end{smallmatrix}\right)
		\left(\begin{smallmatrix}
			n-1 \\ 0
		\end{smallmatrix}\right)
		(-1)^{0}
		&
		\dots
		&
		0
		\\
		\left(\begin{smallmatrix}
			n \\ 0
		\end{smallmatrix}\right)
		\left(\begin{smallmatrix}
			n \\ 0
		\end{smallmatrix}\right)
		(-1)^{0}
		&
		0
		&
		\dots
		&
		0
		
	\end{bmatrix}
	$
	
	$
	(N_{i,j})_{i,j=0}^n =
	\begin{cases}
		\big(
		\begin{smallmatrix}
			n\\
			j
		\end{smallmatrix} 
		\big)
		\big(
		\begin{smallmatrix}
			n-j\\
			j-i-j
		\end{smallmatrix}
		\big)
		(-1)^{n-i-j},  &  i + j \leq n,	\\
		0      & \text{иначе.}
	\end{cases}
	$

	
\end{frame}

\begin{frame}\frametitle{Сплайны из кривых Безье}
	
	\TC{0.5}
	{
		\IG{Screenshot 2024-10-30 190714.jpg}
	}{
		\IG{Screenshot 2024-10-30 190838.jpg}
	}
	
\end{frame}

\section{Би-сплайны}
\frame{\sectionpage}


\begin{frame}\frametitle{Би-сплайны (B-spline)}
	
	$
	\vv C(t) = \sum_{i=0}^{n}N_{i,k}(u)\vv P_i, \ \ u\in [u_\text{min}, u_\text{max}], \ \ k=(1,2,\dots{},n-1).
	$
	
	\vspace{1ex}
	$N_{i,k}=\dfrac{t-u_i}{u_{i+k}-u_i}N_{i,k-1}(u)+
	         \dfrac{u_{i+k}-t}{u_{i+k+1}-u_{i+1}}N_{i+1,k-1}(u)$
	
	\vspace{1ex}
	$U=\{u_0, u_1, \dots  \ u_{n+k+1} \}, \ u_i \leq u_{i+1} $
	
	\vspace{1ex}
	$N_{i,0}(u)=
	\begin{cases}
		1 & u_i\leq t < u_{i+1} \\
		0 & \text{иначе}.
	\end{cases}
	$
	
	
	
	
\end{frame}

 
 \begin{frame}\frametitle{Открытые b-сплайны различного порядка}
 	
 	\TC{0.4}
 	{
 		\includegraphics[page=0]{bspline.pdf}
 	}{
 		
 		Открытый неравномерный вектор узлов:
 		
 		$U=\{u_0, u_1, \ldots, u_{k+n+2}\}$
 		
 				
 		$u_i=
 		\begin{cases}
 			0, & 0 \leq i \leq k-1,\\
 			i-p, & p \leq n,\\
 			n-k+1, & \text{иначе}.
 		\end{cases}$
 		
 		 		

		\vspace{1ex}
 		$n=5$
 		
 		{\color{DarkGoldenrod}$k=5 \quad U = \{ 0,0,0,0,0,0,1,1,1,1,1,1\}$}
 		
 		{\color{Magenta}$k=4 \quad U = \{ 0,0,0,0,0,1,2,2,2,2,2\}$}
 		
 		{\color{Blue}$k=3 \quad U = \{ 0,0,0,0,1,2,3,3,3,3\}$}
 		
 		{\color{ForestGreen}$k=2 \quad U = \{ 0,0,0,1,2,3,4,4,4\}$}
 		
 		{\color{red}$k=1 \quad U = \{ 0,0,1,2,3,4,5,5\}$}
 		
 		
 	}
 	
 \end{frame}
 
   \begin{frame}\frametitle{B-сплайн первой степени}
 	
 	\TC{0.4}
 	{
 		\animategraphics[autoplay,loop, nomouse, palindrome]{15}{Images/\jobname/build_bspline_k1_anim}{0}{114}
 	}{
 		
 		$\vv C(t) = \sum_{i=0}^{1}N_{i,1}(u)\vv P_i,$
 		
 		\includegraphics[page=6,]{bspline.pdf}
 		
 		\includegraphics[page=1,width=\textwidth]{N.pdf}
 		
 	}
 	
 \end{frame}
 
  \begin{frame}\frametitle{B-сплайн второй степени}
 	
 	\TC{0.4}
 	{
 		\animategraphics[autoplay,loop, nomouse, palindrome]{15}{Images/\jobname/build_bspline_k2_anim}{0}{114}
 	}{
 		
 		$\vv C(t) = \sum_{i=0}^{2}N_{i,2}(u)\vv P_i,$
 		
 		\includegraphics[page=5,]{bspline.pdf}
 		
 		\includegraphics[page=2,width=\textwidth]{N.pdf}
 		
 	}
 	
 \end{frame}
 
   \begin{frame}\frametitle{B-сплайн третьей степени}
 	
 	\TC{0.4}
 	{
 		\animategraphics[autoplay,loop, nomouse, palindrome]{15}{Images/\jobname/build_bspline_k3_anim}{0}{114}
 	}{
 		
 		$\vv C(t) = \sum_{i=0}^{3}N_{i,3}(u)\vv P_i,$
 		
 		\includegraphics[page=4,]{bspline.pdf}
 		
 		\includegraphics[page=3,width=\textwidth]{N.pdf}
 		
 	}
 	
 \end{frame}
 
   \begin{frame}\frametitle{B-сплайн четвертой степени}
 	
 	\TC{0.4}
 	{
 		\animategraphics[autoplay,loop, nomouse, palindrome]{15}{Images/\jobname/build_bspline_k4_anim}{0}{114}
 	}{
 		
 		$\vv C(t) = \sum_{i=0}^{4}N_{i,4}(u)\vv P_i,$
 		
 		\includegraphics[page=3,]{bspline.pdf}
 		
 		\includegraphics[page=4,width=\textwidth]{N.pdf}
 		
 	}
 	
 \end{frame}
 
   \begin{frame}\frametitle{B-сплайн пятой степени}
 	
 	\TC{0.4}
 	{
 		\animategraphics[autoplay,loop, nomouse, palindrome]{15}{Images/\jobname/build_bspline_k5_anim}{0}{114}
 	}{
 		
 		$\vv C(t) = \sum_{i=0}^{5}N_{i,5}(u)\vv P_i,$
 		
 		\includegraphics[page=5,]{bspline.pdf}
 		
 		\includegraphics[page=5,width=\textwidth]{N.pdf}
 		
 	}
 	
 \end{frame}
 
 
 \begin{frame}[squeeze]\frametitle{Весовые функции открытых b-сплайнов}
 	
 	\edef\s{0.85}
	\centering
			\includegraphics[page=2,scale=\s]{bspline.pdf}
			\includegraphics[page=3,scale=\s]{bspline.pdf}
			\includegraphics[page=4,scale=\s]{bspline.pdf}
			\includegraphics[page=5,scale=\s]{bspline.pdf}
			\includegraphics[page=6,scale=\s]{bspline.pdf}
			\includegraphics[page=7,scale=\s]{bspline.pdf}

	 {\color{red}$N_{0,k}$} \		 
	{\color{ForestGreen}$N_{1,k}$}\ 		
	{\color{blue}$N_{2,k}$} \		
	{\color{Orange}$N_{3,k}$}\ 		
	{\color{Magenta}$N_{4,k}$}\ 		
	{\color{DeepSkyBlue}$N_{4,k}$}
 
 	\TC{0.5}
 	{
% 		 {$N_{i,4} \quad x = \{ 0,0,0,0,0,0,1,1,1,1,1,1\}$}
% 		
% 		{$N_{i,3} \quad x = \{ 0,0,0,0,0,1,2,2,2,2,2\}$}
% 		
% 		{$N_{i,2} \quad x = \{ 0,0,0,0,1,2,3,3,3,3\}$}
% 		
% 		{$N_{i,1}) \quad x = \{ 0,0,0,1,2,3,4,4,4\}$}
% 		
% 		 {$N_{i,1}) \quad x = \{ 0,0,1,2,3,4,5,5\}$}
 	}{

 	}
 	

 	
 \end{frame}
 

 
\end{document}


