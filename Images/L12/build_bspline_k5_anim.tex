\documentclass[tikz]{standalone}


\usepackage{tikz}
\usepackage[svgnames]{xcolor}
\usepackage{tikz-3dplot}
\usepackage{amsmath} %математические формулы
\usepackage[e]{esvect}  %Красивая стрелочка вектора


\usepackage{polyglossia}
\setdefaultlanguage{russian}
\setotherlanguage{english}
%\setkeys{russian}{babelshorthands=true}
\usepackage{fontspec}
\setmainfont{Times New Roman} [Script=Cyrillic, Mapping=tex-text,]
\setsansfont{Arial} [Script=Cyrillic, Mapping=tex-text,]
\setmonofont{Courier New} [Script=Cyrillic, Mapping=tex-text,]

\usetikzlibrary{calc}
\usetikzlibrary{arrows.meta}
\usetikzlibrary{angles}





\include{../tikzcom.tex}

\begin{document}
	
	\foreach \frame in {0,1,...,99,99,99,99,99,99,99,99,99,99,99,99,99,99,99,99}{
	\begin{tikzpicture}
		
		\clip (0.8,-0.5) rectangle (5.5,5.3);
		\def\dirname{build_bspline_k5_anim}
		
		\input{\dirname/points_tikz.dat}
		\input{\dirname/lp_tikz_frame\frame.dat}
		\input{\dirname/t_tex_frame\frame.dat}
		\input{\dirname/tpoints_latex_frame\frame.dat}
		
		\draw 	(P0) 	node[above right] {$P_0$} --
			(P1) 	node[above] {$P_1$} --
			(P2) 	node[right] {$P_2$} --
			(P3) 	node[right] {$P_3$}--
			(P4) 	node[below] {$P_4$}--
			(P5) 	node[above] {$P_5$};
		
		\fill[red] (LP0) circle(1pt);
		
		\foreach \p in {0,...,5}{
			\fill (P\p) circle(1pt);
		}	
		
		\draw [red] plot file{\dirname/frame\frame.dat};
		
		\foreach \p in \tarr 
		{
			\fill[blue] \p circle(1pt);
			%\node {\tarr};
		}
		
		\pgfkeys{/pgf/number format/.cd,fixed,fixed zerofill,precision=2}
		\draw[red] (LP0) node[right] {$t=\pgfmathprintnumber{\tval}$};
		
	\end{tikzpicture}}
	

	
	
	
	
\end{document}