\documentclass[tikz]{standalone}

\usepackage{tikz}
\usepackage[svgnames]{xcolor}
\usepackage{tikz-3dplot}
\usepackage{amsmath} %математические формулы
\usepackage[e]{esvect}  %Красивая стрелочка вектора


\usepackage{polyglossia}
\setdefaultlanguage{russian}
\setotherlanguage{english}
%\setkeys{russian}{babelshorthands=true}
\usepackage{fontspec}
\setmainfont{Times New Roman} [Script=Cyrillic, Mapping=tex-text,]
\setsansfont{Arial} [Script=Cyrillic, Mapping=tex-text,]
\setmonofont{Courier New} [Script=Cyrillic, Mapping=tex-text,]

\usetikzlibrary{calc}
\usetikzlibrary{arrows.meta}
\usetikzlibrary{angles}





\include{../tikzcom.tex}

\begin{document}
	
	\foreach \frame in {0,1,...,198}{
	\begin{tikzpicture}
		
		%\clip (-1,-0.5) rectangle (4.5,4.5);
		
		\clip (-1.5,-0.8) rectangle (5.5,3.4);
		
		\input{recursive_bezier_anim_n4/tikz.dat}
		\input{recursive_bezier_anim_n4/lp_frame\frame.dat}
		\input{recursive_bezier_anim_n4/p_frame\frame.dat}
		
		\draw 	(P0) 	node[left] {$P_0(\beta^{(0)}_0)$} --
		(P1) 	node[left] {$P_1(\beta^{(0)}_1)$} --
		(P2) 	node[right] {$P_2(\beta^{(0)}_2)$}--
		(P3) 	node[right] {$P_3(\beta^{(0)}_3)$};
		
		
		
		\foreach \p in {0,...,3}{
			\fill (P\p) circle(1pt);
		}	
		
		\draw[very thin, ForestGreen] (PP10) -- (PP11);
		\draw[very thin, ForestGreen] (PP11) -- (PP12);
		\foreach \p in {0,...,2}{
			\fill[ForestGreen] (PP1\p) circle(1pt);
		}	

		
		\draw[very thin, blue] (PP20) -- (PP21);
		\foreach \p in {0,...,1}{
			\fill[blue] (PP2\p) circle(1pt);
		}	
		
		\draw [red] plot file{recursive_bezier_anim_n4/curve_frame\frame.dat};
		\fill[red] (LP0) circle(1pt);
		
		\draw[ForestGreen] (PP10) node[left] {$\beta^{(1)}_0$};
		\draw[ForestGreen] (PP11) node[above] {$\beta^{(1)}_1$};
		\draw[ForestGreen] (PP12) node[right] {$\beta^{(1)}_2$};
		\draw[blue] (PP20) node[below] {$\beta^{(2)}_0$};
		\draw[blue] (PP21) node[below] {$\beta^{(2)}_1$};
		\draw[red] (LP0) node[below] {$\beta^{(3)}_0$};
		
	%	\pgfkeys{/pgf/number format/.cd,fixed,fixed zerofill,precision=2}
	%	\draw (LP0) node[right] {$t=\pgfmathprintnumber{\fpeval{\frame/99}}$};
		
	\end{tikzpicture}}
	

	
	
	
	
\end{document}